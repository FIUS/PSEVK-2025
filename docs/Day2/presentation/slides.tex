\documentclass{../../presentation}

\title{PSE – Vorkurs Tag 2}
\author{Tobias, Philipp, Linus, Tillmann}
\institute{FIUS - Fachgruppe Informatik Universität Stuttgart}
\date{\today}

\makeatletter
\renewcommand{\lecture@pathprefix}[1]{../../logos/}
\makeatother

\usepackage{todonotes}
\setuptodonotes{inline}


\begin{document}

\begin{frame}
  \titlepage
\end{frame}

\begin{frame}
  \listoftodos
\end{frame}

\begin{frame}
  \frametitle{Recap Tag 1}
  \todo{am Anfang immer Vortages recap?}
\end{frame}

%VERZWEIGUNGEN
\begin{frame}
  \frametitle{Was sind Verzweigungen?}
  \begin{itemize}
    \item Programme müssen Entscheidungen treffen \todo{item oder bullet item?}
    \item Beispiel: Links oder rechts gehen?
  \end{itemize}
\end{frame}

%BOOLEAN
\begin{frame}[fragile]
  \frametitle{Boolean – Wahr oder Falsch}
  \begin{itemize}
    \item Datentyp mit zwei Werten: \texttt{true} und \texttt{false}
    \item Wird für Bedingungen verwendet

    \item z.B: \mintinline{java}{heuteDienstag = true;}
    \begin{minted}{java}
boolean heuteDienstag = true;
    \end{minted}
    \item<2-> Wird oft in Bedingungen verwendet, z.B. in if-Anweis
   
    
  \end{itemize}
\end{frame}



%ARITHMETRISCH BOOLEAN
\begin{frame}[fragile]
  \frametitle{Arithmetische Boolean Operatoren}

  \only<1->{Wenn \texttt{a} und \texttt{b} Zahlen sind, prüfen diese Operatoren Beziehungen zwischen ihnen:}

  \begin{itemize}
    \item<1->\texttt{a == b} \quad Wahr, wenn \texttt{a} gleich \texttt{b} ist
    \begin{minted}{java}
boolean result = (a == b); // result true, wenn a gleich b
    \end{minted}

    \item<2->\texttt{a != b} \quad Wahr, wenn \texttt{a} ungleich \texttt{b} ist
    \begin{minted}{java}
boolean result = (a != b); // result true, wenn a ungleich b
    \end{minted}

    \item<3->\texttt{a < b} \quad Wahr, wenn \texttt{a} kleiner als \texttt{b} ist
    \begin{minted}{java}
boolean result = (a < b); // result true, wenn a kleiner als b
    \end{minted}

    \item<4->analog bei \texttt{a > b}, \texttt{a <= b}, \texttt{a >= b} \quad
  \end{itemize}
\end{frame}



%BOOLEAN OPERATOREN
\begin{frame}[fragile]
  \frametitle{Boolean Operatoren}
  \begin{itemize}
    \item<1-> \texttt{!a} \quad \textrightarrow \quad nicht a
    \item<2-> \texttt{a \&\& b}	\quad \textrightarrow \quad a UND b
    \item<3-> \texttt{a || b} \quad \textrightarrow \quad a ODER b
  \end{itemize}
\quad
  \only<4->{Klammern priorisieren:}
    \begin{itemize}
        \item<4-> \mintinline{java}{true || false && false} \quad \textrightarrow \quad \only<5->{\mintinline{java}{true}}
        \item<6-> \mintinline{java}{(true || false) && false} \quad \textrightarrow \quad \only<7->{\mintinline{java}{false}}
    \end{itemize}
\end{frame}



%IF-VERZWEIGUNG
\begin{frame}[fragile]
  \frametitle{if-Verzweigung}

  \begin{itemize}
    \item<1-> Boolescher Ausdruck entscheidet
    \item<1-> Ausführung nur wenn \texttt{true}
    \item<1-> Syntax:
      \begin{minted}[fontsize=\small]{java}
if (Bedingung) {
  // Code bei true
}
      \end{minted}
    \item<2-> Beispiel:
      \begin{minted}[fontsize=\small]{java}
boolean a = true;
if (a) {
  System.out.println("a ist wahr");
}
      \end{minted}
      \begin{ausgabe}
      a ist wahr
      \end{ausgabe}     
  \end{itemize}
\end{frame}



%IF-ELSE-VERZWEIGUNG
\begin{frame}[fragile]

  \begin{itemize}
    \item<1-> Erweiterung der if-Anweisung
    \item<1-> Ausführung bei \texttt{true} oder \texttt{false}
    \item<1-> Syntax:
\begin{minted}[fontsize=\small]{java}
if (Bedingung) {
  // Code bei Bedingnung true
} else {
  // Code bei Bedingung false
}
\end{minted}
    \item<2-> Beispiel:
\begin{minted}[fontsize=\small]{java}
boolean a = false;
if (a) {
  System.out.println("a ist wahr");
} else {
  System.out.println("a ist falsch");
}
\end{minted}
  \end{itemize}
\end{frame}

\begin{frame}[fragile]
    \frametitle{if-else-Verzweigung}
    \begin{itemize}
            \item<1-> Beispiel:
\begin{minted}[fontsize=\small]{java}
boolean a = false;
if (a) {
  System.out.println("a ist wahr");
} else {
  System.out.println("a ist falsch");
}
\end{minted}
\only<1->{%
\begin{ausgabe}
a ist falsch
\end{ausgabe}
}
    \end{itemize}

\end{frame}





%WHILE
\begin{frame}[fragile]
  \frametitle{while-Schleife}

  \begin{itemize}
    \item<1-> Wiederholt Anweisungen, solange eine Bedingung \texttt{true} ist    
    \item<1-> Syntax: \mintinline{java}{while (Bedingung) { /* Code */ }}
    \item<2-> Beispiel:
      \begin{minted}[fontsize=\small]{java}
Scanner scanner = new Scanner(System.in);
String eingabe = "";

while (!eingabe.equals("ok")) {
  System.out.println("Bitte 'ok' eingeben:");
  eingabe = scanner.nextLine();
}
      \end{minted}
      
      \begin{ausgabe}
Bitte 'ok' eingeben:

...

Bitte 'ok' eingeben:

(Benutzer tippt "ok") → Schleife endet

      \end{ausgabe}
    
      \end{itemize}

\end{frame}


%WHILE MIT BREAK
\begin{frame}[fragile]
  \frametitle{break}

  \begin{itemize}
    \item Mit \texttt{break} kann man eine Schleife vorzeitig beenden.
    \item Dies ist nützlich, wenn eine Abbruchbedingung innerhalb der Schleife erkannt wird.
\end{itemize}

\begin{minted}[fontsize=\footnotesize, linenos]{java}
Scanner scanner = new Scanner(System.in);

String passwort = "diegrillung";
String abbruchBedingung = "abbruch";
String momentaneEingabe = "";

while (!momentaneEingabe.equals(passwort)) {
    momentaneEingabe = scanner.nextLine();

    if (momentaneEingabe.equals(abbruchBedingung)) {
        break;
    }
}
System.out.println("I'm in");
\end{minted}
\end{frame}


%CODE TOGHETER
\begin{frame}[plain]
    \centering
    {\Huge\bfseries\textcolor{red}{WURST TOGETHER}}
\end{frame}



%FOR SCHLEIFE
\begin{frame}[fragile]{Was ist eine \texttt{for}-Schleife?}

  \begin{itemize}
    \item Eine \texttt{for}-Schleife hilft dir, etwas \textbf{mehrmals} zu tun.
    \item Syntax:
      \begin{minted}[fontsize=\large]{java}
for (Start; Bedingung; Schritt) {
    // Schleifenrumpf
}
\end{minted}
\item Ablauf:
    \begin{enumerate}
    \item Wo starten wir? (z.B. bei 0)
    \item Bedingung noch erfüllt?
    \item Falls ja, Code im Block ausführen
    \item Zähler ändern i.d.R. einen hochzählen
\end{enumerate}
  \end{itemize}
\todo{bessere Erklärung}
\end{frame}



%FOR BEISPIEL
\begin{frame}[fragile]
  Somit wird aus\dots
  
  \begin{minted}{java}
    int momentaneWurst = 4
int letzteWurst = 8
while (momentaneWurst <= letzteWurst) {
	System.out.println("Schmeiß Wurst Nr." + momentaneWurst + "auf den Grill");
	momentaneWurst++;
}
  \end{minted}
  ganz simpel\dots
  \begin{minted}{java}
      for (int i = 4; i <= 8; i++) {
	System.out.println("Schmeiß Wurst Nr." + i + "auf den Grill");
}
  \end{minted}

\end{frame}

\end{document}
