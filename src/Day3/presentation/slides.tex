\documentclass{../../presentation}

\title{PSE – Vorkurs Tag 2}
\author{Tobias, Philipp, Linus, Tillmann}
\institute{FIUS - Fachgruppe Informatik Universität Stuttgart}
\date{\today}

\makeatletter
\renewcommand{\lecture@pathprefix}[1]{../../logos/}
\makeatother

\usepackage{todonotes}
\setuptodonotes{inline}


\begin{document}

\begin{frame}
    \titlepage
\end{frame}

\begin{frame}
    \listoftodos
\end{frame}

\begin{frame}
    \frametitle{Recap Tag 2}
    \todo{am Anfang immer Vortages recap?}
\end{frame}

\begin{frame}[fragile]
    \frametitle{Funktionen}
    \begin{itemize}
        \item Beim Programmieren benötigt man oft die gleiche Funktionalität an mehreren Stellen.
        \item Bisher musste man dafür den gleichen Code mehrfach schreiben.
        \item Das führt zu:
              \begin{itemize}
                  \item Viel Schreibarbeit
                  \item Fehleranfälligkeit (z.B. Tippfehler, unterschiedliche Änderungen)
                  \item Schwerer wartbarer Code
              \end{itemize}
        \item \textbf{Lösung:} Funktionen ermöglichen Wiederverwendung von Code!
    \end{itemize}
\end{frame}

\begin{frame}
    \frametitle{Funktionen}
    \begin{itemize}
        \item Funktionen sind benannte Codeblöcke, die eine bestimmte Aufgabe erfüllen.
        \item Sie können Parameter entgegennehmen und einen Wert zurückgeben.
        \item Beispiel: Die PSE-Vorkurs Orgas haben sich ordentlich einen hinter die Rüstung gerömert und wollen jeweils wissen, wie viel Promille sie haben.
    \end{itemize}
    \begin{block}{Promille-Berechnung}
        \[
            \text{Promille} \approx \frac{\text{Alkohol in Gramm}}{\text{Körpergewicht in kg} \times 0{,}65}
        \]
        \[
            \text{Alkohol in Gramm} = \text{Getränkemenge in Liter} \times \text{Vol\%} \times 8
        \]
    \end{block}
\end{frame}

\begin{frame}[fragile]
    \frametitle{Funktionen}
    Mithilfe von Funktionen können wir eine oder mehrere Codezeilen auslagern und an verschiedenen Stellen im Code aufrufen.
    \begin{itemize}
        \item Funktionen haben \textbf{Parameter} (= Werte, die der Funktion übergeben werden).
        \item Funktionen besitzen einen \textbf{Rückgabewert}, ähnlich wie mathematische Funktionen, z.\,B.\ $f(x) = x^2$.
        \item Der Rückgabewert wird mit \texttt{return} zurückgegeben; danach wird die Funktion abgebrochen.
    \end{itemize}
    \begin{minted}{java}
Rückgabedatentyp Funktionsbezeichner (Datentyp1 Parametername1, ...) {
    ...
    return Rückgabewert;
}
\end{minted}
\end{frame}

\begin{frame}[fragile]
    \frametitle{Funktionen}
    \begin{itemize}
        \item man kann so viele Parameter angeben, wie man will
        \item Parameter können verschiedene Datentypen haben
        \item Rückgabewert kann auch \texttt{void} sein, wenn die Funktion keinen Wert zurückgibt
    \end{itemize}
    \begin{minted}{java}
    void begruessen(String name) {
        System.out.println("Hallo, " + name + "!");
    }
    \end{minted}

\end{frame}

\begin{frame}[plain]
    \centering
    {\Huge\bfseries{Code together}}
\end{frame}

\end{document}
