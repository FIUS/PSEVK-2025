\documentclass{../../sheet}
\renewcommand{\logopath}{../../logos/}

\title{PSE Vorkurs Tag 3}

\begin{document}
\maketitle
Wir sind so back.

\newpage

\aufgabe{Aufgabe 1: Funktionen Erstellen}

\begin{enumerate}
    \item Erstelle eine Funktion ohne Parameter und ohne Rückgabewert, die den Anfang der Grillung ankündigt:
    \begin{ausgabe}
Es ist Grillereizeit meine Freunde!
    \end{ausgabe}
    \item Erstelle noch eine Funktion, die selbst nichts ausgibt, aber dafür den obigen Text als String zurückgibt. Nutze diese neue Funktion um die gleiche Ausgabe oben zu erstellen. Versuch außerdem die Aufgabe ohne Variable zu lösen.
    \item Erstelle eine dritte Funktion, die einen Namen als Parameter annimmt und die Person herzlich zur Grillerei einläd:
    \begin{ausgabe}
Komm ran Melanie, es gibt Grillung!
    \end{ausgabe}
    \item Erstelle eine Funktion, die prüft ob eine als Parameter übergebene Zahl eine Primzahl ist
    \item Erstelle eine Funktion, die zwei Zahlen addiert, hierfür aber nur den ++ operator verwenden darf (also kein + - * / etc.). 
    \begin{minted}{java}
int a = 3;
a++;
System.out.println(a);
    \end{minted}
    \begin{ausgabe}
        4
    \end{ausgabe}
    \item Nutze die Additionsfunktion um eine neue Funktion zu erstellen, die zwei Zahlen multipliziert, im Körper dieser Funktion dürfen keine Mathematischen Operatoren vorkommen.
    \item Nutze die Multiplikationsfunktion um Fakültät (Fakültät von 5 schreibt man als: 5! = 5*4*3*2*1) zu implementieren.
\end{enumerate}

\end{document}