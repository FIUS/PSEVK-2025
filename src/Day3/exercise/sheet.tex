\documentclass{../../sheet}
\renewcommand{\logopath}{../../logos/}

\title{PSE Vorkurs Tag 3}

\begin{document}
\maketitle
Wir sind so back.

\newpage

\aufgabe{Aufgabe 1: Einfache Funktionen Erstellen}

\begin{enumerate}
    \item Erstelle eine Funktion ohne Parameter und ohne Rückgabewert, die den Anfang der Grillung ankündigt:
          \begin{ausgabe}
              Es ist Grillereizeit meine Freunde!
          \end{ausgabe}
          Hinweis: Wenn man die Funktion nur erstellt dann wird sie noch nicht ausgeführt, sie muss zusätzlich in der main aufgerufen werden.
    \item Erstelle noch eine Funktion, die selbst nichts ausgibt, aber dafür den obigen Text als String zurückgibt. Nutze diese neue Funktion um die gleiche Ausgabe wie oben in Aufgabe 1 zu erstellen. Zusätzlich kannst du versuchen die Aufgabe ohne Variable zu lösen.
    \item Erstelle eine dritte Funktion, die einen Namen als Parameter annimmt und die Person herzlich zur Grillerei einläd:
          \begin{ausgabe}
              Komm ran Melanie, es gibt Grillung!
          \end{ausgabe}
    \item Erstelle eine Funktion, die zwei Zahlen addiert, hierfür aber nur den ++ Operator verwendet (also kein + - * / etc.). Als Erinnerung an den ++ Operator:
          \begin{minted}[linenos=false]{java}
int a = 3;
a++;
System.out.println(a);
    \end{minted}
          \begin{ausgabe}
              4
          \end{ausgabe}
    \item Nutze die Additionsfunktion um eine neue Funktion zu erstellen, die zwei Zahlen multipliziert, im Körper dieser Funktion dürfen keine Mathematischen Operatoren vorkommen.
    \item Nutze die Multiplikationsfunktion um Fakultät zu implementieren.\\
          Hinweis: Beispielsweise schreibt man Fakultät von 6 so: 6! = 6*5*4*3*2*1
\end{enumerate}

\newpage
\aufgabe{Aufgabe 2: Arrays}


\begin{enumerate}
    \item Erstelle ein Array mit 10 \texttt{int}-Werten und Versuche es zu Printen
    \item Das scheint nicht zu klappen, programmiere eine Funktion, die ein Array entgegennimmt und es anschaulich printed.\\
          Hinweis: Erinner dich an Array.length aus der Präsi.
    \item Programmiere eine neue print Funktion, die statt den normalen Listenartigen Arrays 2-Dimensionale Arrays entgegennimmt und wie oben ausgibt.\\
    Hinweis: Ein zweidimensionales Array ist ein Array von Arrays:
    \begin{minted}[linenos=false]{java}
int[][] einArray = new int[3][3]; //ein zweidimensionales Array der Größe 3x3
einArray[0][2] = 3;
System.out.println(einArray[0][2]);
    \end{minted}
    \begin{ausgabe}
        3
    \end{ausgabe}
    \item Programmiere eine Funktion, die eine Seitenlänge als Parameter entgegennimmt und daraus ein 2-D Array aus Einsen mit einem 3 Nullen breiten diagonalen Streifen aus Nullen zurückgibt (siehe Ausgabe). Teste dein Ergebnis mit der Printfunktion die du in Aufgabe 3 programmiert hast.\\
          Seitenlänge 6:
          \begin{ausgabe}
              0 0 0 1 1 1\\
              0 0 0 0 1 1\\
              0 0 0 0 0 1\\
              1 0 0 0 0 0\\
              1 1 0 0 0 0\\
              1 1 1 0 0 0
          \end{ausgabe}
          Seitenlänge 10:
          \begin{ausgabe}
              0 0 0 1 1 1 1 1 1 1 \\
              0 0 0 0 1 1 1 1 1 1 \\
              0 0 0 0 0 1 1 1 1 1 \\
              1 0 0 0 0 0 1 1 1 1 \\
              1 1 0 0 0 0 0 1 1 1 \\
              1 1 1 0 0 0 0 0 1 1 \\
              1 1 1 1 0 0 0 0 0 1 \\
              1 1 1 1 1 0 0 0 0 0 \\
              1 1 1 1 1 1 0 0 0 0 \\
              1 1 1 1 1 1 1 0 0 0
          \end{ausgabe}
          Wenn du an dieser Aufgabe Spaß hattest gibt es im \hyperlink{Aufgabe_4}{Codeanhang} noch mehr Muster die du versuchen kannst zu generieren.
    \item Programmiere eine Funktion, die eine Zahl n entgegennimmt und ein Array mit den ersten n Zahlen der Fibonacci-Reihe zurückgibt. Teste dein Programm mit der Array-Printfunktion die du oben Programmiert hast.\\
          Hinweis: Die Fibonacci Reihe fängt mit 1, 1 an und jedes darauffolgende Element ist die Summe der Beiden vorigen Elemente: 1, 1, 2, 3, 5, 8, 13, ...

\end{enumerate}

\newpage
\aufgabe{HIGHPERFORMER-Aufgabe: Sortieren und Performance-Analyse}
Java bietet eine Funktion um Zeit akkurat zu Messen: \texttt{System.nanoTime()} gibt die Java-interne Zeitmessung in Nanosekunden ($10^{-9}$ Sekunden) aus.
\begin{enumerate}
    \item Miss die durchschnittliche Laufzeit eines prints.\\
    Hinweis: Es ist sinnvoll bei einer Laufzeitanalyse die zu messende Funktion mehrmals durchzuführen und mit dem Durchschnitt der Ergebnisse zu arbeiten.
    \item Erstelle ein Array der Länge 100 und initialisiere es mit zufälligen Werten. \\
          Hinweis: \url{https://docs.oracle.com/javase/8/docs/api/java/util/Random.html}
    \item Überlege dir ein Verfahren zum sortieren von Arrays und programmiere eine Funktion, die das zufällige Array aus Aufgabe 2. als Parameter entgegennimmt und es sortiert zurückgibt. Miss die durchschnittliche Laufzeit deiner Funktion.          
    \item Versuche deine Funktion zu optimieren und somit schneller zu machen. Wir (die PSE-Hauptorgas) haben unkreativ wie wir sind einen bekannten Sortieralgorithmus (Quicksort) implementiert, vielleicht bekommst du es hin unsere Zeit zu schlagen. Unser Algorithmus ist im \hyperlink{Highperformer}{Codeanhang} damit du dessen Laufzeit selbst testen kannst.\\
          Hinweis: Hierbei kann es sehr hilfreich sein nicht nur die gesamte Laufzeit zu analysieren, sondern auch einzelne Teile des Codes, um zu wissen wo man am meisten Zeit einsparen kann.
    \item Informier dich im Internet über unterschiedliche Sortieralgorithmen und versuche einen der dir gefällt zu implementieren. Natürlich gehört da auch wieder eine Laufzeitanalyse dazu.
\end{enumerate}

\newpage
\aufgabe{Codeanhang}
\hypertarget{Aufgabe_4}{}
Aufgabe 2: Arrays:\\
4. Alle mit Seitenlänge 13, aber deine Funktion sollte die Arrays natürlich je nach Eingabegröße generieren.
\begin{ausgabe}
    1 0 1 0 1 0 1 0 1 0 1 0 1 \\
    0 1 0 1 0 1 0 1 0 1 0 1 0 \\
    1 0 1 0 1 0 1 0 1 0 1 0 1 \\
    0 1 0 1 0 1 0 1 0 1 0 1 0 \\
    1 0 1 0 1 0 1 0 1 0 1 0 1 \\
    0 1 0 1 0 1 0 1 0 1 0 1 0 \\
    1 0 1 0 1 0 1 0 1 0 1 0 1 \\
    0 1 0 1 0 1 0 1 0 1 0 1 0 \\
    1 0 1 0 1 0 1 0 1 0 1 0 1 \\
    0 1 0 1 0 1 0 1 0 1 0 1 0 \\
    1 0 1 0 1 0 1 0 1 0 1 0 1 \\
    0 1 0 1 0 1 0 1 0 1 0 1 0 \\
    1 0 1 0 1 0 1 0 1 0 1 0 1
\end{ausgabe}
\begin{ausgabe}
    1 1 1 1 1 1 0 1 1 1 1 1 1 \\
    1 1 1 1 1 0 0 0 1 1 1 1 1 \\
    1 1 1 1 0 0 0 0 0 1 1 1 1 \\
    1 1 1 0 0 0 0 0 0 0 1 1 1 \\
    1 1 0 0 0 0 0 0 0 0 0 1 1 \\
    1 0 0 0 0 0 0 0 0 0 0 0 1 \\
    0 0 0 0 0 0 0 0 0 0 0 0 0 \\
    1 0 0 0 0 0 0 0 0 0 0 0 1 \\
    1 1 0 0 0 0 0 0 0 0 0 1 1 \\
    1 1 1 0 0 0 0 0 0 0 1 1 1 \\
    1 1 1 1 0 0 0 0 0 1 1 1 1 \\
    1 1 1 1 1 0 0 0 1 1 1 1 1 \\
    1 1 1 1 1 1 0 1 1 1 1 1 1
\end{ausgabe}
\begin{ausgabe}
    1 1 1 1 1 1 1 1 1 1 1 1 1 \\
    1 1 0 1 1 1 0 1 1 1 0 1 1 \\
    1 0 0 0 1 0 0 0 1 0 0 0 1 \\
    1 1 0 1 1 1 0 1 1 1 0 1 1 \\
    1 1 1 1 1 1 1 1 1 1 1 1 1 \\
    1 1 0 1 1 1 0 1 1 1 0 1 1 \\
    1 0 0 0 1 0 0 0 1 0 0 0 1 \\
    1 1 0 1 1 1 0 1 1 1 0 1 1 \\
    1 1 1 1 1 1 1 1 1 1 1 1 1 \\
    1 1 0 1 1 1 0 1 1 1 0 1 1 \\
    1 0 0 0 1 0 0 0 1 0 0 0 1 \\
    1 1 0 1 1 1 0 1 1 1 0 1 1 \\
    1 1 1 1 1 1 1 1 1 1 1 1 1
\end{ausgabe}

Higherformer-Aufgabe:\\
3.
\hypertarget{Highperformer}{}
\begin{minted}[linenos=false]{java}
public int[] quicksort(int[]){
    //TODO
}
\end{minted}
\end{document}