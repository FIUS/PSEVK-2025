\documentclass{../../presentation}

\title{PSE - Vorkurs}
\author{Tobias, Philipp, Linus, Tillmann}
\institute{FIUS - Fachgruppe Informatik Universität Stuttgart}
\date{\today}

\makeatletter
\renewcommand{\lecture@pathprefix}[1]{../../logos/}
\makeatother

\usepackage{todonotes}
\setuptodonotes{inline}

\begin{document}

\begin{frame}
  \titlepage
\end{frame}

\begin{frame}
  \listoftodos
\end{frame}

\section{Introduction}

\begin{frame}[fragile]
  \frametitle{WLAN}
  Die Universität bietet ein Gast WLAN an jedoch wollen wir in das offizielle WLAN der Universität einsteigen.
  \newline
  \begin{itemize}
    \item iOS und Android
    \item Windows
    \item MacOS
    \item Linux
  \end{itemize}
\end{frame}

\begin{frame}[fragile]
  \frametitle{WLAN - iOS und Android}
  \begin{itemize}
    \item Im AppStore bzw. Playstore nach \enquote{geteduroam} suchen und die App installieren
    \item Dann nach Universität Stuttgart suchen
    \item Nun Student wählen und eure Daten eingeben
    \item \texttt{stxxxxxx@stud.uni-stuttgart.de} - Benutzername
    \item Und eurer Passwort eingeben
    \item Nun auf Verbinden klicken und fertig
  \end{itemize}
\end{frame}

\begin{frame}[fragile]
  \frametitle{WLAN - Windows}
  \begin{itemize}
    \item Im Internet auf \href{https://geteduroam.app}{\enquote{geteduroam.app}} gehen und die Windows Version herunterladen
    \item Dann nach Universität Stuttgart suchen
    \item Nun Student wählen und eure Daten eingeben
    \item Nun auf Verbinden klicken und fertig
  \end{itemize}
\end{frame}

\begin{frame}[fragile]
  \frametitle{WLAN - MacOS}
  \begin{itemize}
    \item Im Internet auf \href{http://uni-stuttgart.de/eduroam}{\enquote{uni-stuttgart.de/eduroam}} gehen
    \item und als Gruppe \enquote{Student} wählen
    \item Lädet das Konfigurationspaket herunter
    \item In euren Systemeinstellungen auf \enquote{Geräteverwaltung} gehen
    \item Nun über das Plus-Symbol das Konfigurationspaket hinzufügen
    \item \texttt{stxxxxx} - Benutzername
    \item Und euer Passwort eingeben
    \item Nun auf Verbinden klicken und fertig
  \end{itemize}
\end{frame}

\begin{frame}[fragile]
  \frametitle{WLAN - Linux}
  \begin{itemize}
    \item Im Internet auf \href{http://uni-stuttgart.de/eduroam}{\enquote{uni-stuttgart.de/eduroam}} gehen
    \item Als Benutzergruppe \enquote{Student} wählen
    \item Unten links auf \enquote{einen anderen Installer auswählen} klicken
    \item \enquote{Linux} auswählen
    \item Über den Button das Pythonskript herunterladen
    \item Im Terminal mit \texttt{\$ python3 [Dateipfad zum Pythonskript]} ausführen
    \item Den Anweisungen folgen und Benutzername (\texttt{stxxxxxx@stud.uni-stuttgart.de}) sowie Passwort eingeben
  \end{itemize}
\end{frame}

\begin{frame}
  \frametitle{IntelliJ IDEA}
  \textbf{\achtung{} Wir werden uns Lediglich auf IntelliJ IDEA konzentrieren, wenn ihr eine andere IDE nutzen wollt, können wir keinen Support garantieren. }
  \newline
  \begin{itemize}
    \item Wir laden uns erst die JetBrains Toolbox herunter, um die IDE zu installieren
    \item Dann installieren wir IntelliJ IDEA Community Edition
  \end{itemize}
\end{frame}

\begin{frame}
  \includegraphics[width=1\linewidth]{img/memeIDE.jpg}
\end{frame}

\begin{frame}
  \frametitle{IntelliJ IDEA - Installation}
  \begin{tikzpicture}[remember picture,overlay]
    \node[at=(current page.center), anchor=center, yshift=-2] {\includegraphics[width=1\linewidth]{img/toolBoxDownload.png}};
    \node[at=(current page.north), anchor=north, align=center, yshift=-1.2cm] {Geht auf \href{https://www.jetbrains.com/de-de/toolbox-app/}{\texttt{\enquote{jetbrains.com/toolbox-app/}}} und ladet die Toolbox herunter.};
  \end{tikzpicture}
\end{frame}

\begin{frame}
  \frametitle{IntelliJ IDEA - Installation}
  \begin{tikzpicture}[remember picture,overlay]
    \node[at=(current page.east), anchor=east, yshift=-2] {\includegraphics[width=.35\linewidth]{img/IntelliJDownload.png}};
    \node[at=(current page.west), anchor=west, align=center, yshift=0cm, xshift=2cm, text width=8cm] {Nach der Installation der Toolbox öffnet sich ein Fenster, in dem ihr die IDE auswählen könnt, die ihr installieren wollt. Wählt \enquote{IntelliJ IDEA Community Edition} aus und klickt auf \enquote{Install}. \newline \todo{Besser machen}};
  \end{tikzpicture}
\end{frame}

\begin{frame}
  \frametitle{JDK - Installation}
  Wir benötigen ein JDK (Java Development Kit), um Java Programme zu schreiben und auszuführen.
  \newline
  \begin{itemize}
    \item \achtung{} MacOS: auf File/ Datei gehen oben links in der Menüleiste
    \item Nun auf \texttt{Project Structure} klicken
    \item Dann auf \texttt{SDKs} klicken
    \item Dann auf das Plus-Symbol klicken und auf \texttt{Download JDK} klicken
  \end{itemize}
\end{frame}

\begin{frame}
  \frametitle{JDK - Installation}
  \achtung{} Die Universität nutzt die OpenJDK Version 21, die wir auch verwenden werden.
  \begin{itemize}
    \item Dann Microsoft OpenJDK als Anbieter/ Vendor auswählen
    \item Und nen geeigneten Speicher Ort auswählen, der Standart Pfad sollte aber ausreichen.
  \end{itemize}
\end{frame}

\begin{frame}
  \frametitle{Projekt erstellen}
  \begin{itemize}
    \item In IntelliJ auf neues Projekt erstellen/ New Project klicken
    \item Java auswählen und das Projekt firstProjekt nennen
    \item Nun nen geeigneten Speicher Ort auswählen \achtung{} KEIN Git Reposotory erstellen
    \item Als Build System IntelliJ verwenden und als JDK unsere derzeitig herruntergeladene 21 auswählen
    \item Wir wollen keinen Sample Code
  \end{itemize}
\end{frame}

\begin{frame}
  \frametitle{Projekt erstellen}
  Es sollte nun so bei euch aussehen:

  \vspace{1cm}

  \includegraphics[width=1\linewidth]{img/projektCreation.png}
\end{frame}

\begin{frame}
  \frametitle{Projekt erstellen}
  \begin{itemize}
    \item Nun auf \texttt{Create} klicken und das Projekt wird erstellt
    \item Nun links in der Projektansicht auf \texttt{src} rechtsklicken und eine neue Java Klasse erstellen
    \item Diese Klasse \texttt{Main} nennen und auf \texttt{Enter} drücken
    \item Nun sollte sich eine neue Datei öffnen, in der wir unseren Code schreiben
  \end{itemize}
\end{frame}

\begin{frame}[fragile]
  \frametitle{Erste Java Klasse}
  Nun schreiben wir folgenen Code in die Main Klasse rein
  \begin{minted}{java}
    public class Main {
    public static void main(String[] args) {
		
      }
    }
  \end{minted}
  \achtung{} Was das alles bedeutet werden wir euch dann in den nächsten Tagen erklären!
\end{frame}

\begin{frame}
  \frametitle{GitHub}
  GitHub ist eine Plattform, auf der man Code online speichern, verwalten und mit anderen teilen kann. Es basiert auf Git, einem Versionskontrollsystem.
\end{frame}

\begin{frame}
  \frametitle{Typische Begriffe auf GitHub}
  \todo{Neu machen ist ultra hässlich}
  \begin{itemize}
    \item \textbf{Repository (Repo):} Dein Projektordner auf GitHub.
    \item \textbf{Commit:} Eine gespeicherte Änderung am Code.
    \item \textbf{Push:} Hochladen deiner Änderungen auf GitHub.
    \item \textbf{Pull:} Änderungen von GitHub herunterladen.
    \item \textbf{Clone:} Ein Projekt von GitHub auf deinen Computer kopieren.
  \end{itemize}
\end{frame}

\begin{frame}
  \frametitle{Warum GitHub für Java-Projekte?}
  \todo{Neu machen ist ultra hässlich}
  \begin{itemize}
    \item \textbf{Sicherung:} Du kannst deinen Java-Code (Klassen, Methoden, Projekte) sicher speichern.
    \item \textbf{Versionskontrolle:} Du siehst, was sich zwischen Versionen geändert hat – praktisch beim Lernen oder bei Fehlern.
    \item \textbf{Zusammenarbeit:} Du kannst mit anderen zusammenarbeiten (Teamarbeit, Feedback).
    \item \textbf{Überall verfügbar:} Zugriff von überall – egal ob Uni, zuhause oder unterwegs.
  \end{itemize}
\end{frame}

\begin{frame}[fragile]
  \frametitle{Variablen}
  \begin{itemize}
    \item Wir benötigen Variablen um Werte im Programm zwischenzuspeichern
    \item Variablen haben einen Namen (Bezeichner), einen Datentyp und speichern einen Wert
    \item Über den Namen kann später auf den gespeicherten Wert zugegriffen werden
          \begin{minted}{java}
      int alter = 20;
    \end{minted}
    \item Hier ist \texttt{alter} der Name der Variable, \texttt{int} der Datentyp (ganzzahlige Zahl) und \texttt{20} der gespeicherte Wert
  \end{itemize}
  \achtung{} Bezeichner müssen mit einem Buchstaben, Unterstrich (\_) oder Dollarzeichen (\$) beginnen.
  Danach dürfen beliebig viele Buchstaben, Ziffern, Unterstriche oder Dollarzeichen folgen -- aber keine Leerzeichen oder Sonderzeichen.
\end{frame}

\begin{frame}[fragile]
  \frametitle{Variablen}
  \begin{itemize}
    \item Deklaration \quad \textrightarrow \quad Variable wird angelegt/erstellt \textbf{OHNE} einen Wert zuzuweisen
          \begin{minted}{java}
      int zahl;
    \end{minted}
          \pause
    \item Initialisierung \quad \textrightarrow \quad Variable wird sein \textbf{ERSTER} Wert zugewiesen
          \begin{minted}{java}
      zahl = 42;
    \end{minted}
          \pause
  \end{itemize}
  \begin{minted}{java}
  //In der Regel Deklaration und Initialisierung gleichzeitig:
  int zahl = 42;

  //aber auch so möglich
  int zahl;
  zahl = 42;
  \end{minted}
\end{frame}

\begin{frame}[fragile]
  \frametitle{Variablen}
  \begin{itemize}
    \item Variablen können auch später im Programm verändert werden
          \begin{minted}{java}
      int alter = 20;
      alter = 21; // alter ist jetzt 21
    \end{minted}
    \item Variablen können auch mehrfach deklariert werden, aber nur einmal initialisiert
          \begin{minted}{java}
      int a = 5;
      int b = 10;
      int c = a + b; // c ist jetzt 15
    \end{minted}
  \end{itemize}
  \achtung{} Variablen müssen vor ihrer Verwendung deklariert und initialisiert werden
\end{frame}

\end{document}
