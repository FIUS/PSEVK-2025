\documentclass{../../presentation}

\title{PSE - Vorkurs Tag 1}
\author{Linus, Philipp, Tillmann, Tobias}
\institute{FIUS - Fachgruppe Informatik Universität Stuttgart}
\date{06.09.2025}

\makeatletter
\renewcommand{\lecture@pathprefix}[1]{../../logos/}
\makeatother

\usepackage{todonotes}
\setuptodonotes{inline}

\begin{document}

\begin{frame}
  \titlepage
\end{frame}

\begin{frame}
  \listoftodos
\end{frame}

\section{Introduction}

\begin{frame}
  \frametitle{To-Do: Entwicklungsumgebung einrichten}

  \begin{itemize}
    \item Verbinde dich mit dem WLAN
    \item Installiere IntelliJ IDEA (Community Edition) \newline
          \achtung{} Ihr könnt auch eine andere IDE verwenden
    \item Installiere das JDK (Java Development Kit)
    \item Projekt erstellen
  \end{itemize}
  \vfill
  \centering
  \Large
  Bei Fragen einfach melden! \\
  \vspace{0.5em}
  Danach geht's gleich weiter...
\end{frame}




\begin{frame}[fragile]
  \frametitle{Erste Java Klasse}
  Nun schreiben wir folgenen Code in die Main Klasse rein
  \begin{minted}{java}
    public class Main {
      public static void main(String[] args) {
        System.out.println("Hallo Welt!");
      }
    }
  \end{minted}
  \achtung{} Was das alles bedeutet werden wir euch dann in den nächsten Tagen erklären!
\end{frame}

\begin{frame}
  \frametitle{Erste Java Klasse}
  \begin{itemize}
    \item Das Programm soll nun bei allen von euch gestartet werden
    \item Dazu klickt auf das grüne Dreieck oben rechts in der IDE
    \item Wenn alles korrekt eingerichtet ist, sollte die Ausgabe "Hallo Welt!'' erscheinen.
          \achtung{} Wenn eine Fehlermeldung erscheint, überprüft bitte den Code alternativ fragt einen Tutor.
  \end{itemize}
\end{frame}

\begin{frame}
  \frametitle{GitHub}
  GitHub ist eine Plattform, auf der man Code online speichern, verwalten und mit anderen teilen kann. Es basiert auf Git, einem Versionskontrollsystem.
\end{frame}

\begin{frame}
  \frametitle{Warum GitHub für Java-Projekte?}
  \pause
  \begin{itemize}
    \item \textbf{Sicherung:} Java-Code (Klassen, Methoden, Projekte) kann sicher gespeichert werden.
          \pause
    \item \textbf{Versionskontrolle:} Änderungen zwischen Versionen sind nachvollziehbar – praktisch beim Lernen oder bei Fehlern.
          \pause
    \item \textbf{Zusammenarbeit:} Gemeinsames Arbeiten an Projekten und Feedback sind möglich.
          \pause
    \item \textbf{Überall verfügbar:} Zugriff von verschiedenen Orten – beispielsweise an der Uni, zuhause oder unterwegs.
  \end{itemize}
\end{frame}

\begin{frame}[fragile]
  \frametitle{Variablen}
  \pause
  \begin{itemize}
    \item Variablen werden benötigt, um Werte im Programm zwischenzuspeichern
          \pause
    \item Sie haben einen Namen (Bezeichner), einen Datentyp und speichern einen Wert
          \pause
    \item Über den Namen kann später auf den gespeicherten Wert zugegriffen werden
          \pause
          \begin{minted}{java}
      int alter = 20;
    \end{minted}
    \item \texttt{alter} ist der Name der Variable, \texttt{int} der Datentyp (ganzzahlige Zahl) und \texttt{20} der gespeicherte Wert
  \end{itemize}
\end{frame}

\begin{frame}[fragile]
  \frametitle{Konzept \texttt{null}}
  \pause
  \begin{itemize}
    \item \texttt{null} steht für „kein Wert“
          \begin{minted}{java}
      String name = null; // name zeigt auf nichts
      \end{minted}
          \pause
    \item Zugriff auf Methoden oder Eigenschaften einer \texttt{null}-Referenz führt zu einem Fehler "\color{red}\texttt{NullPointerException}\color{black}".
  \end{itemize}
  \pause
  \achtung{} \texttt{null} ist nicht dasselbe wie eine leere Zeichenkette \texttt{''} oder die Zahl 0.
\end{frame}

\begin{frame}[fragile]
  \frametitle{Variablen}
  \pause
  \begin{itemize}
    \item Deklaration \quad \textrightarrow \quad Variable wird erstellt \textbf{OHNE} einen Wert zuzuweisen
          \begin{minted}{java}
      int zahl;
    \end{minted}
          \pause
    \item Initialisierung \quad \textrightarrow \quad Variable wird sein \textbf{ERSTER} Wert zugewiesen
          \begin{minted}{java}
      zahl = 42;
    \end{minted}
          \pause
  \end{itemize}
  \begin{minted}{java}
  //In der Regel Deklaration und Initialisierung gleichzeitig
  int zahl = 42;

  //aber auch so möglich
  int zahl;
  // ...
  zahl = 42;
  \end{minted}
\end{frame}

\begin{frame}[fragile]
  \frametitle{Variablen}
  \pause
  \begin{itemize}
    \item Variablen können auch später im Programm verändert werden
          \begin{minted}{java}
      int alter = 20;
      alter = 21; // alter ist jetzt 21
    \end{minted}
  \end{itemize}
\end{frame}

\begin{frame}
  \frametitle{Datentypen in Java}
  \pause
  \centering
  \rowcolors{2}{tablerow}{white}
  \begin{tabular}{l l}
    \rowcolor{tablehead}
    \textbf{Typ} & \textbf{Beschreibung}                          \\
    int          & Ganzzahlen (z.\,B. 1, 2, 3)                    \\
    float        & Kommazahlen (weniger genau)                    \\
    double       & Kommazahlen (hohe Genauigkeit)                 \\
    char         & Einzelnes Zeichen (z.\,B. 'a')                 \\
    short        & Kleine Ganzzahlen                              \\
    long         & Große Ganzzahlen                               \\
    boolean      & Wahrheitswert (\texttt{true} / \texttt{false}) \\
    String       & Text (z.\,B. "Hallo")                          \\
  \end{tabular}
\end{frame}


\begin{frame}
  \frametitle{Operatoren in Java}
  \pause
  \centering
  \rowcolors{2}{tablerow}{white}
  \begin{tabular}{l l l}
    \rowcolor{tablehead}
    \textbf{Operator} & \textbf{Bedeutung}    & \textbf{Gültige Typen} \\
    +                 & Addition / Verkettung & Zahlen, String         \\
    -                 & Subtraktion           & Zahlen                 \\
    *                 & Multiplikation        & Zahlen                 \\
    /                 & Division              & Zahlen                 \\
    \%                & Modulo (Rest)         & Ganzzahlen             \\
  \end{tabular}
\end{frame}

\begin{frame}[fragile]
  \frametitle{Beispiel: \texttt{String + Zahl}}
  \pause
  \begin{minted}{java}
String text = "Ergebnis: ";
int zahl = 5;
System.out.println(text + zahl);
  \end{minted}
  \pause
  \begin{ausgabe}
    Ergebnis: 5
  \end{ausgabe}
\end{frame}

\begin{frame}[fragile]
  \frametitle{Kommentare in Java}
  \pause
  \begin{itemize}
    \item \texttt{// Einzeilige Kommentare}
    \item \texttt{/* Mehrzeilige Kommentare */}
  \end{itemize}
  \pause
  \vspace{1em}
  \begin{minted}{java}
float gewicht = 23.4; // Gewicht in kg

/*
Berechnet die Summe
zweier Zahlen
*/
int summe = x + y;
  \end{minted}
\end{frame}

\begin{frame}[fragile]
  \frametitle{Textausgabe in Java}
  \pause
  Problem: Wir wollen irgendwas in der Konsole ausgeben
  \pause
  \vspace{0.5em}
  \begin{minted}{java}
System.out.print("Das ist eine Konsolen Ausgabe");
  \end{minted}
  \begin{ausgabe}
    Das ist eine Konsolen Ausgabe
  \end{ausgabe}
  \pause
  \begin{minted}{java}
System.out.print("Das ist eine Konsolen Ausgabe");
System.out.print("Das ist noch eine Ausgabe");
  \end{minted}
  \pause
  \begin{ausgabe}
    Das ist eine Konsolen AusgabeDas ist noch eine Ausgabe
  \end{ausgabe}
\end{frame}

\begin{frame}[fragile]
  \frametitle{Zeilenumbruch: \texttt{println} und \texttt{\textbackslash n}}
  \pause
  \begin{minted}{java}
System.out.println("1. Ausgabe");
System.out.println("2. Ausgabe");
  \end{minted}
  \pause
  \begin{ausgabe}
    1. Ausgabe \newline
    2. Ausgabe
  \end{ausgabe}
  \pause
  \begin{minted}{java}
System.out.print("1. Ausgabe\n");
System.out.print("2. Ausgabe");
  \end{minted}
  \pause
  \begin{ausgabe}
    1. Ausgabe \newline
    2. Ausgabe
  \end{ausgabe}
\end{frame}

\begin{frame}[fragile]
  \frametitle{Ausgabe von Variablen}
  \pause
  Man kann nicht nur Text, sondern auch Variablen ausgeben:
  \pause
  \begin{minted}{java}
int number = 42;
System.out.println(number);
  \end{minted}
  \begin{ausgabe}
    42
  \end{ausgabe}
  \pause
  Oder beides kombiniert:
  \begin{minted}{java}
int number = 42;
System.out.println("Meine Lieblingszahl ist " + number + "!");
  \end{minted}
  \pause
  \begin{ausgabe}
    Meine Lieblingszahl ist 42!
  \end{ausgabe}
\end{frame}

\begin{frame}[plain]
  \centering
  {\Huge\bfseries{fuiz time!!}}
\end{frame}

\begin{frame}[fragile]
  \frametitle{Frage 1: Was wird ausgegeben?}
  \begin{minted}{java}
    System.out.println(4 + 5);
  \end{minted}
  \pause
  \begin{ausgabe}
    9
  \end{ausgabe}
\end{frame}

\begin{frame}[fragile]
  \frametitle{Frage 2: Was wird ausgegeben?}
  \begin{minted}{java}
    System.out.println(3 + "4");
  \end{minted}
  \pause
  \begin{ausgabe}
    34
  \end{ausgabe}
\end{frame}

\begin{frame}[fragile]
  \frametitle{Frage 3: Was wird ausgegeben?}
  \begin{minted}{java}
    System.out.println(2);
    System.out.println(3);
  \end{minted}
  \pause
  \begin{ausgabe}
    23
  \end{ausgabe}
\end{frame}

\begin{frame}[fragile]
  \frametitle{Frage 4:Was wird ausgegeben?}
  \begin{minted}{java}
    System.out.println(13 % 3);
  \end{minted}
  \pause
  \begin{ausgabe}
    1
  \end{ausgabe}
\end{frame}

\begin{frame}[fragile]
  \frametitle{Frage 5:Was wird ausgegeben?}
  \begin{minted}{java}
    System.out.println(10 + ((3 * 4)+ "7"));
  \end{minted}
  \pause
  \begin{ausgabe}
    10127
  \end{ausgabe}
\end{frame}

\begin{frame}[fragile]
  \frametitle{Frage 6: Was wird ausgegeben?}
  \begin{minted}{java}
    int a = 3;
    // a = 3 + 4;
    System.out.println(a);
  \end{minted}
  \pause
  \begin{ausgabe}
    3
  \end{ausgabe}
\end{frame}

\begin{frame}[fragile]
  \frametitle{Frage 7:Was wird ausgegeben?}
  \begin{minted}{java}
    System.out.println(10 + "6" * 4);
  \end{minted}
  \pause
  \begin{ausgabe}
    \textcolor{red}{Fehler}
  \end{ausgabe}
\end{frame}

\begin{frame}[fragile]
  \frametitle{Frage 8:Was wird ausgegeben?}
  \begin{minted}{java}
    int a = 8 * 4;
    a = a / 5;
    System.out.println(a);
  \end{minted}
  \pause
  \begin{ausgabe}
    6
  \end{ausgabe}
\end{frame}

\begin{frame}[fragile]
  \frametitle{Frage 9: Was wird ausgegeben?}
  \begin{minted}{java}
    float a = 8 * 4;
    a = a / 5;
    System.out.println(a + "3");
  \end{minted}
  \pause
  \begin{ausgabe}
    6.43
  \end{ausgabe}
\end{frame}

\begin{frame}[fragile]
  \frametitle{Eingabe mit \texttt{Scanner}}
  \pause
  Benutzereingaben werden in Java mit der \texttt{Scanner} Klasse verarbeitet
  \pause
  \begin{itemize}
    \item Mit \texttt{Scanner} kann man Benutzereingaben aus der Konsole einlesen.
          \pause
    \item Dafür muss die Klasse \texttt{Scanner} aus dem Paket \texttt{java.util} importiert werden.
          \pause
    \item Imports steht ganz oben im Java-Quelltext
  \end{itemize}
  \pause
  \begin{minted}{java}
import java.util.Scanner;
\end{minted}
  \pause
  \begin{minted}{java}
Scanner scanner = new Scanner(System.in);
String name = scanner.nextLine();
System.out.println(name);
  \end{minted}
\end{frame}

\begin{frame}[fragile]
  \frametitle{Syntaxfehler - Beispiele}
  \pause
  \begin{itemize}
    \item \textbf{Syntaxfehler:} Code verletzt Java-Regeln (z.\,B. fehlendes `;`)
  \end{itemize}
  \begin{minted}{java}
// Fehlendes Semikolon
System.out.println("Hallo")

// Variable nicht deklariert
zahl = 5;
  \end{minted}
\end{frame}

\begin{frame}[fragile]
  \frametitle{Laufzeitfehler - Beispiele}
  \begin{itemize}
    \item \textbf{Laufzeitfehler:} Fehler während der Ausführung (z.\,B. Division durch 0)
  \end{itemize}
  \begin{minted}{java}
String text = null;
text.length(); // NullPointerException

int zahl = 5 / 0; // ArithmeticException
  \end{minted}
\end{frame}

% \begin{frame}[fragile]
%   \todo{Logikfehler in Tag 3}
%   \frametitle{Logikfehler - Beispiel}
%   \begin{itemize}
%     \item \textbf{Logikfehler:} Programm läuft, aber Ergebnis ist falsch (z.\,B. falsche Berechnung)
%   \end{itemize}
%   \begin{minted}{java}
% public int add(int x, int y) {
%   return x - y; // Falsche Operation!
% }
%   \end{minted}
% \end{frame}

\begin{frame}[fragile]
  \frametitle{Fehler suche IDE}
  \pause
  \begin{itemize}
    \item IntelliJ IDEA bietet viele Tools zur Fehlersuche
          \pause
    \item Fehler werden rot unterstrichen und in der rechten Leiste angezeigt
          \pause
    \item Wenn ihr über den Fehler hovert, seht ihr eine Beschreibung des Problems
          \pause
    \item Nutze die integrierte Konsole, um Ausgaben und Fehler zu sehen \newline
  \end{itemize}
  \begin{minted}{java}
  System.out.println("Lecker Grillerei!")
  \end{minted}
  \pause
  \begin{ausgabe}
    \color{red}java: ';' expected
  \end{ausgabe}
\end{frame}

\begin{frame}[fragile]
  \frametitle{Hinweise zum Programmieren}
  \pause
  \begin{itemize}
    \item Achte auf korrekte Syntax (Semikolons, Klammern, etc.)
          \pause
    \item Verwende sprechende Variablennamen (z.\,B. \texttt{alter} statt \texttt{x})
          \pause
    \item Teste deinen Code regelmäßig, um Fehler frühzeitig zu finden
  \end{itemize}
\end{frame}

\begin{frame}
  \frametitle{Autoformatter in IntelliJ IDEA}
  \pause
  \begin{itemize}
    \item Autoformatter sorgt für übersichtliche und einheitliche Code-Formatierung.
          \pause
    \item Tastenkombination: \texttt{Strg + Alt + L} (Windows/Linux), \texttt{Cmd + Opt + L} (Mac).
          \pause
    \item Achtet auf Einrückungen, Abstände und Zeilenumbrüche.
          \pause
    \item Gut formatierter Code ist leichter lesbar und verständlich.
          \pause
    \item Tipp: Code regelmäßig vor dem Speichern oder Teilen formatieren.
  \end{itemize}
\end{frame}

\end{document}