% Dieses sheet hat nur die Aufgaben, das heißt es beginnt erst nach der inhaltlichen Präsi, wo schon alle Internet und intelliJ haben sollten.
\documentclass{../../sheet}
\renewcommand{\logopath}{../../logos/}

\title{PSE Vorkurs Tag 1}

\begin{document}
\maketitle

\aufgabe{Wichtige Shortcuts: (Windows/Linux)}
\begin{itemize}
    \item Code automatisch formattieren (z.B. richtig Einrücken): \textbf{Strg+Alt+L}
    \item Code kopieren: \textbf{Strg+C}
    \item Den mit \textbf{Strg+C} kopierten Code woanders einfügen: \textbf{Strg+V}
    \item Letzte Aktion rückgängig machen: \textbf{Strg+Z}
    \item IntelliJ Autovervollständigungsvorschläge annehmen: \textbf{Tab}
\end{itemize}

\aufgabe{Wichtige Shortcuts: (macOS)}
\begin{itemize}
    \item Code automatisch formattieren (z.B. richtig Einrücken): \textbf{cmd+option+L}
    \item Code kopieren: \textbf{cmd+C}
    \item Den mit \textbf{cmd+C} kopierten Code woanders einfügen: \textbf{cmd+V}
    \item Letzte Aktion rückgängig machen: \textbf{cmd+Z}
    \item IntelliJ Autovervollständigungsvorschläge annehmen: \textbf{Tab}
\end{itemize}

\aufgabe{Hilfe nichts funktioniert :(}
Manche von euch haben sicher schon Erfahrung mit Programmieren, wenn ihr auf ein Problem stoßt, fragt einfach mal eure Sitznachbahrn und helft euch gegenseitig. Ihr könnt natürlich auch die freundlichen Tutoren ranholen und um Hilfe bitten.
\newpage


\aufgabe{Aufgabe 1: Variablen und Ausgaben}
Man kann mit \texttt{System.out.println(einString)} den Wert von \texttt{einString} in der Konsole (unten in der Mitte bei der Ausführung des Codes) ausgeben. Folgender code:
\begin{minted}[linenos=false]{java}
System.out.println("PSE-Zeit ist Grillzeit")
\end{minted}
Gibt folgendes aus:
\begin{ausgabe}
    PSE-Zeit ist Grillzeit
\end{ausgabe}

\begin{enumerate}
    \item Kopiert folgenden Grill und nutzt print(s) um ihn in der Konsole auszugeben.\\
    Hinweis: Kopiert den Grill direkt zwischen die Gänsefüsschen in euer \texttt{System.out.println("grill hier")},  
          \begin{ausgabe}
              .............................................................................
              .............................................................................
              ...................................@.........................................
              ..................................\#@...-.....................................
              ..................................@@....@....................................
              .................................@@@...@@....................................
              .................................@@....@@....................................
              ................................@@-...@@.....................................
              ................................@@...@@-.....................................
              ................................@:...@@......................................
              ................................@....@.......................................
              .................................=....\%......................................
              .............................................................................
              ...................@@@@@@@@@@@@@@@@@@@@@@@@@@@@@@@@@@@.......................
              ....................@@@@@@@@@@@@@@@@@@@@@@@@@@@@@@@@@........................
              ....................\%@@@@@@@@@@@@@@@@@@@@@@@@@@@@@@@@@@@@....................
              .....................@@@..@@@@@@@@@@@@@@@@@@@@@@@@@@@@@@.....................
              .....................:@@@*.@@@@@@@@@@@@@@@@@@@@@@@@*.........................
              ......................-@@@@@@@@@@@@@@@@@@@@@@@@@@@\#..........................
              ........................@@@@@@@@@@@@@@@@@@@@@@@@@............................
              ..........................@@@@@@@@@@@@@@@@@@@@@-.............................
              ............................@@@@@@@@@@@@@@@@@................................
              ...............................@@@@@@@@@@@:..................................
              ..............................@@@-......@@@..................................
              .............................@@@*........@@@.................................
              ............................@@@@..........@@@................................
              ...........................@@@@............@@@...............................
              ......................\#@@@@@@@.............:@@@..............................
              ....................@@@@@@@@@@..............+@@@.............................
              ...................=@@@@..@@@@@@@@@@@@@@@@@@@@@@@............................
              ...................@@@@....@@@@.:.............@@@*...........................
              ....................@@@@@@@@@@\#................@@@:..........................
              .....................@@@@@@@@...................@@@..........................
              .............................................................................
          \end{ausgabe}
    \item Erstelle eine Variable mit beliebigem Wert und gib sie aus, ändere im selben Programmdurchlauf den Wert und gib dieselbe Variable nochmal aus.
    \item Erstelle zwei int Variablen und gib die Summe und das Produkt aus.
    \item Erstelle eine Variable mit deinem Namen und schreibe genau ein Print, das dich mit deinem Namen begrüßt.
\end{enumerate}

\newpage
\aufgabe{Aufgabe 2: Scanner}
Man kann während das Program ausgeführt wird auch Werte aus der Konsole einlesen. Das geht mit einem Scanner Objekt. Mit \texttt{scannerObjekt.next()} wird die nächste eingegebene Zeile eingelesen. Mit \textbf{Enter} beendet man die Eingabe.
\begin{minted}[linenos=false]{java}
Scanner derScanner = new Scanner(System.in);
System.out.println("Gib deinen Namen ein: ");
String deinName = derScanner.next();
System.out.println("Du heißt: " + deinName);
\end{minted}

\begin{enumerate}
    \item Programmiere einen addierer, der zwei Zahlen einliest und die Summe ausgibt. Etwas funktioniert nicht, überlegt mal warum (Tipp im \hyperlink{Aufgabe_2.1}{Codeanhang})?
\begin{ausgabe}
Gib eine Zahl ein: 12\\
Gib noch eine Zahl ein: 3\\
Die Summe ist: 15
\end{ausgabe}
    \item Programmiere ein Modulo Übungsprogramm, bei dem man zwei Zahlen mit dem Scanner eingibt, dann das Ergebnis von \texttt{zahl1 \% zahl2} eingeben soll und dann am Ende das korrekte Ergebnis ausgegeben wird.
\begin{ausgabe}
Gib eine Zahl ein: 14\\
Gib noch eine Zahl ein: 3\\
Gib dein Ergebnis ein: 6\\
Das richtige Ergebnis von 14 \% 3 ist: 2
\end{ausgabe}
    \item Schreibe eine kurze Geschichte (oder such dir einen kurzen Text im Internet). Markiere dann ein paar Wörter mit ihren Wortarten und lass den Nutzer sie mit dem Scanner durch andere ersetzen. Gib dann die Geschichte mit den vom Nutzer eingegebenen Wörtern aus. Zum Beispiel wird: ''\textit{Die \underline{PSE}-Vorkurs Hauptorgas lieben \underline{grillen}.}'' \space zu:
          \begin{ausgabe}
              Gib ein Thema ein: backen\\
              Gib ein Verb ein: Ofen sprengen\\
              Die backen-Vorkurs Hauptorgas lieben Ofen sprengen.
          \end{ausgabe}
          Lass deine Sitznachbarn und/oder die Helfer dein Programm ausführen ohne, dass sie wissen welcher Text dahintersteckt.
\end{enumerate}

\newpage
\aufgabe{HIGHPERFORMER-Aufgabe: Zufallszahlen}
Java stellt einen Zufallszahlengenerator zur Verfügung. Dieser besitzt unter anderem folgende Funktionen:

\begin{minted}[linenos=false]{java}
Random rndGenerator = new Random();

//gibt ein zufälligen float zwischen 0.0 und 1.0 zurück:
float zufallsFloat = rndGenerator.nextFloat();

//gibt ein zufälligen int zwischen 0 und 9 zurück (10 wird als Obergrenze ausgeschlossen)
int zufallsInt = rndGenerator.nextInt(10);
\end{minted}
Die Offizielle Dokumentation für Random findet ihr hier: \url{https://docs.oracle.com/javase/8/docs/api/java/util/Random.html}

\begin{enumerate}
    \item Modifiziere dein Moduloübungsprogramm aus Aufgabe 2.2 so, dass man die beiden Zahlen nicht mehr selbst eingibt (war auch wirklich zu leicht so), sondern so, dass diese zufällig generiert werden.
    \item Kopiert das Programm aus dem \hyperlink{Aufgabe_2.2}{Codeanhang}. Dort bekommt ihr einen sogenannten Seed (5-Stellig) vom eingebauten Java Zufallsgenerator. Generiert mit den mathematischen Operatoren die ihr jetzt kennt anhand dieses Seeds eine neue 10-Stellige Zufallszahl (ihr dürft natürlich nicht die \texttt{Random} Klasse verwenden). Wir (die PSE-Vorkurs Hauptorgas) haben ein Testerprogramm geschrieben, dass versucht eure Lösung zu bewerten und eventuelle Schwächen eures Ansatzes aufzuzeigen, doch vielleicht kriegt ihr es auch hin den Test auszutricksen.
\end{enumerate}

\newpage
\aufgabe{Codeanhang}
\hypertarget{Aufgabe_2.1}{}
\textbf{Aufgabe 2.1}\\
\texttt{derScanner.next()} gibt einen String zurück, wenn man die beiden addiert werden die beiden Zahlen stumpf hintereinandergeschrieben, statt gescheit addiert zu werden. Um stattdessen einen \texttt{int} einzulesen besitzt Scanner die \texttt{nextInt()} Funktion.\\
\textbf{Highperformer Aufgabe 2:}
\hypertarget{Aufgabe_2.2}{}
\textbf{Highperformer Aufgabe 2:}
\\TODO die das hier vom Git klonen lassen statt kopieren
\begin{minted}[linenos=false]{java}
public class main{
    public int zufallsZahl(int seed) {
        //seed ist die ursprüngliche 5-stellige Zahl mit der ihr rechnen könnt
        System.out.println(seed);
        int rueckgabeZahl;

        // Schreibt euren code, der am ende eure neue zehnstellige Zahl
        // der Variable rueckgabeZahl zuweisen soll zwischen hier...


        // ... und hier
        return rueckgabeZahl
    }

    public static void main(string[] args){
        int anzahlTestZahlen = 100;
        int[] zahlen = new int[anzahlTestZahlen];
        Random rnd = new Random()

        for (int i = 0; i < anzahlTestZahlen; i++) {
            int seed = rnd.nextInt(100000)
            zahlen[i] = zufallsZahl(seed)
        }

        analyse(zahlen);
    }

    public void analyse(int[] zufallsZahlen) {
        //TODO
    }
}
\end{minted}
\end{document}