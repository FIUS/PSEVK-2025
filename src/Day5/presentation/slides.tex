\documentclass{../../presentation}

\title{PSE - Vorkurs Tag 5}
\author{Linus, Philipp, Tillmann, Tobias}
\institute{FIUS - Fachgruppe Informatik Universität Stuttgart}
\date{10.09.2025}

\makeatletter
\renewcommand{\lecture@pathprefix}[1]{../../logos/}
\makeatother

\usepackage{todonotes}
\setuptodonotes{inline}


\begin{document}

\begin{frame}
	\titlepage
\end{frame}

\begin{frame}[fragile]
	\frametitle{Recap Tag 4}
	\pause
	\begin{minted}{java}
public class Student {
    String name;
    int matrikelnummer;

    void lernen() {
	System.out.println(name + " lernt fleißig!");
    }
}
	\end{minted}
\end{frame}

\begin{frame}[fragile]
	\frametitle{Recap Tag 4}
	\pause
	\begin{minted}{java}
public Student(String name, int matrikelnummer) {
	this.name = name;
	this.matrikelnummer = matrikelnummer;
}

Student paul = new Student("Paul", 12345);
paul.lernen(); // → Paul lernt fleißig!
	\end{minted}
	\pause
	\begin{minted}{java}
Student anna = new Student("Anna", 67890);

System.out.println(anna.matrikelnummer); // → 67890
anna.lernen(); // → Anna lernt fleißig!
	\end{minted}
\end{frame}

\begin{frame}[fragile]
	\frametitle{Recap Tag 4}
	\pause
	\begin{minted}{java}
public class Student {
private final String name;
public int matrikelnummer;

public Student(String name, int matrikelnummer) {
	this.name = name;
	this.matrikelnummer = matrikelnummer;
}
}
	\end{minted}
\end{frame}

\begin{frame}[fragile]
	\frametitle{Map (Dictionary)}
	\begin{itemize}
		\item Eine Map (oder ein Dictionary) ist eine Datenstruktur, die Schlüssel-Wert-Paare speichert.
		\item Jeder Schlüssel ist eindeutig und wird verwendet, um den entsprechenden Wert zu finden.
		\item Maps sind nützlich, um Daten zu organisieren und schnell auf sie zuzugreifen.
	\end{itemize}
\end{frame}

\begin{frame}[fragile]
	\frametitle{Map (Dictionary)}
	\begin{itemize}
		\item Wir wollen Werte mit einem \texttt{Schlüssel} verbinden → z. B. “Name” → Matrikelnummer
		\item Dafür gibt es in Java Map-Strukturen - der Standard: HashMap
	\end{itemize}
	\begin{minted}{java}
import java.util.HashMap;
	\end{minted}
	\begin{minted}{java}
HashMap<String, Integer> studis = new HashMap<>();
studis.put("Melanie", 1976370);
studis.put("Paul", 1249609);

System.out.println(studis.get("Melanie"));
	\end{minted}
	\begin{ausgabe}
		1976370
	\end{ausgabe}
\end{frame}

\begin{frame}
	\frametitle{Map - Methoden}
	\begin{table}[h]
		\centering
		\rowcolors{2}{tablerow}{white}
		\begin{tabular}{l l}
			\rowcolor{tablehead}
			\textbf{Methode}        & \textbf{Beschreibung}                    \\
			\texttt{put(k, v)}      & Fügt einen neuen Schlüssel-Wert-Paar ein \\
			\texttt{get(k)}         & Gibt den Wert zu einem Schlüssel zurück  \\
			\texttt{remove(k)}      & Entfernt einen Eintrag                   \\
			\texttt{containsKey(k)} & Prüft, ob Schlüssel existiert            \\
			\texttt{keySet()}       & Gibt alle Schlüssel zurück               \\
			\texttt{values()}       & Gibt alle Werte zurück                   \\
			\texttt{size()}         & Gibt die Anzahl der Einträge zurück      \\
		\end{tabular}
	\end{table}
\end{frame}

\begin{frame}[fragile]
	\frametitle{Map - remove()}
	\begin{itemize}
		\item Die \texttt{remove()} Methode entfernt einen Eintrag aus der Map
		\item Gibt den entfernten Wert zurück oder \texttt{null}, falls der Schlüssel nicht existiert
	\end{itemize}
	\begin{minted}{java}
HashMap<String, Integer> studis = new HashMap<>();
studis.put("Melanie", 1976370);
studis.put("Paul", 1249609);

System.out.println("Vor remove: " + studis.size());
Integer removed = studis.remove("Paul");
System.out.println("Entfernter Wert: " + removed);
System.out.println("Nach remove: " + studis.size());
	\end{minted}
	\begin{ausgabe}
		Vor remove: 2 \newline
		Entfernter Wert: 1249609 \newline
		Nach remove: 1
	\end{ausgabe}
\end{frame}

\begin{frame}[fragile]
	\frametitle{Map - containsKey()}
	\begin{itemize}
		\item Die \texttt{containsKey()} Methode prüft, ob ein bestimmter Schlüssel in der Map existiert
		\item Gibt \texttt{true} zurück, wenn der Schlüssel vorhanden ist, sonst \texttt{false}
	\end{itemize}
	\begin{minted}{java}
HashMap<String, Integer> studis = new HashMap<>();
studis.put("Melanie", 1976370);

System.out.println("Paul existiert: " + studis.containsKey("Paul"));

if (studis.containsKey("Melanie")) {
    System.out.println("Melanies Matrikelnummer: " + studis.get("Melanie"));
}
	\end{minted}
	\begin{ausgabe}
		Paul existiert: false \newline
		Melanies Matrikelnummer: 1976370
	\end{ausgabe}
\end{frame}

\begin{frame}[fragile]
	\frametitle{Map - keySet()}
	\begin{itemize}
		\item Die \texttt{keySet()} Methode gibt alle Schlüssel der Map als Set zurück
		\item Nützlich zum Iterieren über alle Einträge der Map
	\end{itemize}
	\begin{minted}{java}
HashMap<String, Integer> studis = new HashMap<>();
studis.put("Melanie", 1976370);
studis.put("Paul", 1249609);

System.out.print("Alle Namen: ");
for (String name : studis.keySet()) {
    System.out.print(name + ", ");
}
	\end{minted}
	\begin{ausgabe}
		Alle Namen: Paul, Melanie,
	\end{ausgabe}
\end{frame}

\begin{frame}[fragile]
	\frametitle{Map - values()}
	\begin{itemize}
		\item Die \texttt{values()} Methode gibt alle Werte der Map als Collection zurück
		\item Ermöglicht den Zugriff auf alle gespeicherten Werte ohne die Schlüssel
	\end{itemize}
	\begin{minted}{java}
HashMap<String, Integer> studis = new HashMap<>();
studis.put("Melanie", 1976370);
studis.put("Paul", 1249609);

System.out.println("Alle Matrikelnummern: ");
for (Integer number : studis.values()) {
    System.out.println(number + ", ");
}
	\end{minted}
	\begin{ausgabe}
		Alle Matrikelnummern: 1249609, 1976370,
	\end{ausgabe}
\end{frame}

\begin{frame}[fragile]
    \frametitle{Ende}
    \begin{columns}
        \begin{column}{0.65\textwidth}
            \begin{itemize}
                \item Folien und Aufgaben: \newline \url{https://fius.de/index.php/pse-vk-folien/}
            \end{itemize}
        \end{column}
        \begin{column}{0.35\textwidth}
            \qrcode{https://fius.de/index.php/pse-vk-folien/}
        \end{column}
    \end{columns}

    \vspace{3em}

    \begin{columns}
        \begin{column}{0.65\textwidth}
            \begin{itemize}
                \item Feedback: \newline \url{\feedbackurl}
            \end{itemize}
        \end{column}
        \begin{column}{0.35\textwidth}
            \qrcode{\feedbackurl}
        \end{column}
    \end{columns}

    Wenn ihr Fragen habt, sagt Bescheid!

\end{frame}

\end{document}