\documentclass{../../presentation}

\title{PSE – Vorkurs Tag 5}
\author{Tobias, Philipp, Linus, Tillmann}
\institute{FIUS - Fachgruppe Informatik Universität Stuttgart}
\date{\today}

\makeatletter
\renewcommand{\lecture@pathprefix}[1]{../../logos/}
\makeatother

\usepackage{todonotes}
\setuptodonotes{inline}


\begin{document}

\begin{frame}
	\titlepage
\end{frame}

\begin{frame}
	\listoftodos
\end{frame}

\begin{frame}
	\frametitle{Recap Tag 4}
	\todo{am Anfang immer Vortages recap?}
\end{frame}

\begin{frame}[fragile]
\frametitle{Map (Dictionary)}
\begin{itemize}
	\item Eine Map (oder ein Dictionary) ist eine Datenstruktur, die Schlüssel-Wert-Paare speichert.
	\item Jeder Schlüssel ist eindeutig und wird verwendet, um den entsprechenden Wert zu finden.
	\item Maps sind nützlich, um Daten zu organisieren und schnell auf sie zuzugreifen.
\end{itemize}
\end{frame}

\end{document}