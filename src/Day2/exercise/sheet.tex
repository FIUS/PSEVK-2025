\documentclass{../../sheet}
\renewcommand{\logopath}{../../logos/}

\title{PSE Vorkurs Tag 2}

\begin{document}
\maketitle

\textbf{Hilfe suchen und finden}\\
Beim Programmieren sind Google und Dokumentationen der Sprache/Umgebung die ihr nutzt sehr hilfreich. Wenn ihr ein Problem habt, kann es neben Tutoren fragen auch sinnvoll sein einfach mal eine Fehlermeldung oder eine neue Funktion im Internet zu suchen. Die Offizielle Dokumentation für Java ist hier: \url{https://docs.oracle.com/javase/8/docs/api/}.

\newpage
\aufgabe{Aufgabe 1: Boole'sche Ausdrücke und Verzweigungen}
\begin{enumerate}
    \item Schreibe für die UNO diesen Freitag (Party zum Start des Wintersemesters) ein Programm, bei dem das Alter der Studis geprüft wird und ausgegeben wird ob die Person auf die Party gelassen wird oder nicht. Frage mit dem Scanner vom Nutzer das Alter ab. Erinnerung an die Scanner Syntax:\\ \url{https://docs.oracle.com/javase/8/docs/api/java/util/Scanner.html}
    \begin{ausgabe}
        Wie alt bist du: 17\\
        Sorry, du darfst noch nicht rein :/
    \end{ausgabe}
    \item Füge dem Programm ein Preiskalkulierer hinzu, der mit dem Scanner einliest ob die Person Ersti ist oder nicht (Tipp: lies einen boolean Wert ein). Bei der UNO zahlen Erstis nämlich nur 3 Euro, alle anderen aber 7 Euro. Vergiss aber nicht, dass ja alle unter 18 leider nicht reindürfen und deswegen keinen Preis ausgegeben bekommen sollten.
    \begin{ausgabe}
        Bist du ein Ersti? true\\
        Für dich sinds 3 Euro
    \end{ausgabe}
    \item Jetzt fehlen nur noch Drinks, frage ab wie viel Geld die Person dabei hat und jeweils wieviele Shots (1 Euro), Longdrinks (3 Euro), Softdrinks (2 Euro), Cocktails (3 Euro) und Biere (2 Euro) die Person will. Gib am Ende aus ob das Geld ausreicht.
    \begin{ausgabe}
        Wie viel Geld hast du? 23\\
        Wie viele Shots hättest du gerne? 1\\
        Wie viele Longdrinks hättest du gerne? 3\\
        Wie viele Softdrinks hättest du gerne? 6\\
        Wie viele Cocktails hättest du gerne? 3\\
        Wie viele Biere hättest du gerne? 4\\
        Das sind insgesamt 38 Euro\\
        Du hast zu wenig Geld
    \end{ausgabe}
\end{enumerate}

\newpage
\aufgabe{Aufgabe 2: Schleifen}
\begin{enumerate}
    \item Wir arbeiten weiter mit dem UNO Programm. Implementiere mithilfe von for-Schleifen, dass jedes Getränk Nummeriert ausgegeben wird. Das soll natürlich nur passieren wenn man die Getränke auch bezahlen kann:
          \begin{ausgabe}
              Shot 1 ist  eingeschenkt\\
              Shot 2 ist  eingeschenkt\\
              Shot 3 ist  eingeschenkt\\
              Cocktail 1 ist  Gemixt\\
              Bier 1 ist geöffnet\\
              Bier 2 ist geöffnet
          \end{ausgabe}
    \item Da Studis ja bekanntlichermaßen nicht rechnen können gehen wir davon aus, dass es oft passiert das eine Person mehr Shots will als sie sich leisten kann. Nutze eine While Schleife um solange neue Getränkeanfragen einzulesen, bis die Person sich ihre Bestellung leisten kann.
    \begin{ausgabe}
        Wie viel Geld hast du? 23\\
        Wie viele Shots hättest du gerne? 1\\
        Wie viele Longdrinks hättest du gerne? 3\\
        Wie viele Softdrinks hättest du gerne? 6\\
        Wie viele Cocktails hättest du gerne? 3\\
        Wie viele Biere hättest du gerne? 4\\
        Das sind insgesamt 38 Euro\\
        Du hast zu wenig Geld, probier eine andere Bestellung\\
        Wie viele Shots hättest du gerne? 1\\
        ...
    \end{ausgabe}
    \item Schreibe ein Programm, dass vom Nutzer eine Zahl einliest und sie Rückwärts geschrieben wieder ausgibt. Nutze keine String Variablen.\\
    Hinweis: Es kann Sinnvoll sein Methoden aus dem \texttt{Math} Package zu verwenden, z.B. \texttt{Math.log(eineZahl)}. Die Dokumentation dafür findet ihr hier: \url{https://docs.oracle.com/javase/8/docs/api/java/lang/Math.html}
    \item Schreibe ein Programm, dass vom Nutzer eine Zahl einliest und prüft ob diese ein Palindrom ist (Vorwärts und Rückwärts gelesen gleich, z.b.: 4165614 oder 829928). Nutze natürlich wieder keine String Variablen.
\end{enumerate}

\newpage
\aufgabe{HIGHPERFORMER-Aufgabe: Formeln und Erfüllbarkeit}
In der Highperformer Aufgabe heute geht es um die Erfüllbarkeit von Formeln. Eine Formel ist ein Boole'scher Ausdruck. 
\\\\
Man unterteilt Formen in immer Wahr (Tautologien), immer Falsch sind (Unerfüllbar) oder je nach den Werten der Variablen manchmal Wahr und manchmal Falsch (Erfüllbar):
\\\\
\textbf{Erfüllbar: } \texttt{((a \&\& c) || b)}, da sie wahr wird wenn z.b. \texttt{b = true} ist, jedoch keine Tautologie, da die Gesamtformel auch falsch sein kann, bspw. mit \texttt{a = false}, \texttt{c = true} und \texttt{b = false}\\
\textbf{Unerfüllbar: } \texttt{(a \&\& !b \&\& !a \&\& c)}, da a zugleich wahr und falsch sein müsste\\
\textbf{Tautologie: } \texttt{(!(a \&\& b) == !a || !b)}, da die linke Seite nur \texttt{false} ist, wenn a und b beide \texttt{true} sind, für die Rechte seite gilt genau das gleiche. Deswegen ist sind die beiden Seiten immer gleich.
\begin{enumerate}
    \item Prüfe folgende Boole'sche Ausdrücke Systematisch mit einem Programm auf ihre Erfüllbarkeit und gib dann erfüllbar, unerfüllbar oder Tautologie aus.
          \begin{enumerate}
              \item \begin{ausgabe} \texttt{(!(a \&\& b) || a) \&\& (!(c \&\& d) || c) \&\& (e || !e)} \end{ausgabe} %Tautologie
              \item \begin{ausgabe} \texttt{((a \&\& b \&\& c) \&\& !(a || (b || c))) \&\& d} \end{ausgabe} %Unerfüllbar
              \item \begin{ausgabe} \texttt{((a \&\& !b \&\& !c) || (!a \&\& b \&\& d)) \&\& (e || f)} \end{ausgabe} %Erfüllbar 3/16
              \item \begin{ausgabe} \texttt{((a \&\& b) || (!a || b) || (!b)) \&\&
                            ((c \&\& d) || (!c || d) || (!d)) \&\&
                            ((e \&\& f) || (!e || f) || (!f)) \&\&
                            ((g \&\& h) || (!g || h) || (!h))}
                    \end{ausgabe} %Tautologie
              \item \begin{ausgabe} \texttt{((a \&\& b \&\& !c) \&\& (!a || !b || c)) \&\&
                            ((d \&\& !e \&\& f) \&\& (!d || e || !f)) \&\&
                            ((g \&\& h) \&\& (!g \&\& !h))}
                    \end{ausgabe} %Unerfüllbar
              \item \begin{ausgabe} \texttt{(!(a \&\& b \&\& c) || (a \&\& (b || c))) || (d || !d)} \end{ausgabe} %Tautologie
              \item \begin{ausgabe} \texttt{((a \&\& b \&\& !c \&\& !d \&\& !e) ||
                            (!a \&\& c \&\& d \&\& !b \&\& !e) ||
                            (!b \&\& !c \&\& !d \&\& e \&\& f)) \&\& (!g || h)}
                    \end{ausgabe} %Erfüllbar 9/128
              \item \begin{ausgabe} \texttt{(!w || x) \&\&
                            (k || !j) \&\&
                            (!z || !a) \&\&
                            (!q || r) \&\&
                            (!k || l) \&\&\\
                            (!v || a) \&\&
                            (!g || h) \&\&
                            (!h || i) \&\&
                            (!u || v) \&\&
                            (!a || b) \&\&\\
                            (j || !k) \&\&
                            (!l || m) \&\&
                            (!n || o) \&\&
                            (!c || d) \&\&
                            (!x || y) \&\&\\
                            (b || f) \&\&
                            (!b || c) \&\&
                            (y || !z) \&\&
                            (!t || u) \&\&
                            (!m || n) \&\&\\
                            (!o || p) \&\&
                            (!r || s) \&\&
                            (!j || k) \&\&
                            (!s || t) \&\&
                            (!i || j) \&\&\\
                            (d || e) \&\&
                            (!d || e) \&\&
                            (z || a) \&\&
                            (!f || g) \&\&
                            (!e || f) \&\&\\
                            (a || !b) \&\&
                            (c || !d) \&\&
                            (!p || q) \&\&
                            (!x || d) \&\&
                            (!w || x) \&\&\\
                            (h || !e) \&\&
                            (!n || q) \&\&
                            (!v || a) \&\&
                            (!s || h) \&\&\\
                            (u || !t) \&\&
                            (!g || h)}
                    \end{ausgabe} %Unerfüllbar
                    Tipp: Da wegen den und's alle Klammern wahr werden müssen, kann man z.b. aus \texttt{(a || b)} schließen, dass wenn \texttt{a = false}, dann muss \texttt{b = true} sein und andersrum.\\
          \end{enumerate}
\end{enumerate}
\end{document}
