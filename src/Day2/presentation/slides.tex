\documentclass{../../presentation}

\title{PSE - Vorkurs Tag 2}
\author{Linus, Philipp, Tillmann, Tobias}
\institute{FIUS - Fachgruppe Informatik Universität Stuttgart}
\date{07.09.2025}

\makeatletter
\renewcommand{\lecture@pathprefix}[1]{../../logos/}
\makeatother

\setlength{\marginparwidth}{2cm}
\usepackage{todonotes}
\setuptodonotes{inline}


\begin{document}

\begin{frame}
  \titlepage
\end{frame}

\begin{frame}[fragile]
  \frametitle{Recap Tag 1}
  \pause
  \begin{center}
    \begin{minted}{java}
int alter = 20;    // Deklaration + Initialisierung
int zahl;          // Deklaration
zahl = 42;         // Initialisierung
    \end{minted}

    \rowcolors{2}{tablerow}{white}
    \pause
    \begin{tabular}{l l l}
      \rowcolor{tablehead}
      \textbf{Operator} & \textbf{Bedeutung}           & \textbf{Beispiele}              \\
      \texttt{+}        & Addition / String-Verkettung & 3\texttt{+}2, "a"\texttt{+}"b"  \\
      \texttt{-}        & Subtraktion                  & 7\texttt{-}5                    \\
      \texttt{*}        & Multiplikation               & 3\texttt{*}4                    \\
      \texttt{/}        & Division                     & 10\texttt{/}2                   \\
      \texttt{\%}       & Modulo                       & 5\texttt{\%}2                   \\
    \end{tabular}
  \end{center}
\end{frame}


\begin{frame}[fragile]
  \frametitle{Recap Tag 1}
  \pause
  \begin{minted}{java}
System.out.print("Hallo");
System.out.println("Welt");
System.out.println("Zahl: " + 42);
  \end{minted}
  \pause
  \begin{minted}{java}
import java.util.Scanner;

Scanner sc = new Scanner(System.in);
int x = sc.nextInt();
int y = sc.nextInt();
System.out.println(x + y);
  \end{minted}
\end{frame}


%BOOLEAN
\begin{frame}[fragile]
  \frametitle{\texttt{Boolean} - Wahr oder Falsch}
  \pause
  \begin{itemize}
    \item Datentyp mit zwei Werten: \texttt{true} oder \texttt{false}
          \pause
    \item Wahrheitswerte speichern und verarbeiten
          \pause
    \item z.B:
          \begin{minted}{java}
boolean heuteDienstag = true;
    \end{minted}

  \end{itemize}
\end{frame}



%ARITHMETRISCH BOOLEAN
\begin{frame}[fragile]
  \frametitle{\texttt{Boolean} Operatoren}
  \pause
  \only<1->{Wenn \texttt{a} und \texttt{b} Werte sind, prüfen diese Operatoren Beziehungen zwischen ihnen:}

  \begin{itemize}
    \item<2->\texttt{heuteDienstag == esRegnet} \quad Wahr, wenn \texttt{heuteDienstag} gleich \texttt{esRegnet} ist
          \begin{minted}{java}
boolean result = (heuteDienstag == esRegnet); // result true, wenn heuteDienstag gleich esRegnet
    \end{minted}
    \achtung für Strings: \texttt{equals()} benutzen!

  \end{itemize}
\end{frame}

\begin{frame}[fragile]
  \frametitle{\texttt{Boolean} Operatoren}
  \pause

  \begin{itemize}
    \item<1->\texttt{heuteDienstag != esRegnet} \quad Wahr, wenn \texttt{heuteDienstag} ungleich \texttt{esRegnet} ist
          \begin{minted}{java}
boolean result = (heuteDienstag != esRegnet); // result true, wenn heuteDienstag ungleich esRegnet
    \end{minted}

    \item<3->\texttt{preis < geld} \quad Wahr, wenn \texttt{geld} kleiner als \texttt{preis} ist
          \begin{minted}{java}
boolean result = (preis < geld); // result true, wenn preis kleiner als geld
    \end{minted}

    \item<4->analog bei \texttt{preis > geld}, \texttt{preis <= geld}, \texttt{preis >= geld} \quad
  \end{itemize}
\end{frame}


%BOOLEAN OPERATOREN
\begin{frame}[fragile]
  \frametitle{\texttt{Boolean} Operatoren}
  \pause
  \begin{columns}
    %Linke Spalte: Operatoren und Wahrheitstabellen
    \begin{column}{0.5\textwidth}
      \begin{itemize}
        \item \texttt{!a} \quad \textrightarrow \quad NICHT a
              \pause
        \item \texttt{a \&\& b} \quad \textrightarrow \quad a UND b\\[0.3em]
              \pause
              {\rowcolors{2}{tablerow}{white}
                \begin{tabular}{l l l}
                  \rowcolor{tablehead}
                  \textbf{a}     & \textbf{b}     & \textbf{a \&\& b} \\
                  \texttt{true}  & \texttt{true}  & \texttt{true}     \\
                  \texttt{true}  & \texttt{false} & \texttt{false}    \\
                  \texttt{false} & \texttt{true}  & \texttt{false}    \\
                  \texttt{false} & \texttt{false} & \texttt{false}    \\
                \end{tabular}
              }
      \end{itemize}
    \end{column}

    \begin{column}{0.5\textwidth}
      \newline
      \pause
      \begin{itemize}
        \item \texttt{a || b} \quad \textrightarrow \quad a ODER b\\[0.3em]
              \pause
              {\rowcolors{2}{tablerow}{white}
                \begin{tabular}{l l l}
                  \rowcolor{tablehead}
                  \textbf{a}     & \textbf{b}     & \textbf{a || b} \\
                  \texttt{true}  & \texttt{true}  & \texttt{true}   \\
                  \texttt{true}  & \texttt{false} & \texttt{true}   \\
                  \texttt{false} & \texttt{true}  & \texttt{true}   \\
                  \texttt{false} & \texttt{false} & \texttt{false}  \\
                \end{tabular}
              }
      \end{itemize}
    \end{column}
  \end{columns}
  \vspace{0.5cm}
  \pause
  \textbf{Klammern priorisieren:}
  \begin{minted}{java}
  true || false && false 
  // -> true
  (true || false) && false
  // -> false
  \end{minted}
\end{frame}

%IF-VERZWEIGUNG
\begin{frame}[fragile]
  \frametitle{\texttt{if}-Verzweigung}
  \pause
  \begin{itemize}
    \item Ausführung nur wenn Bedingung \texttt{true}
          \pause
          \begin{minted}{java}
if (Bedingung) {
  // Code bei true
}
      \end{minted}
          \pause
    \item Beispiel:
          \begin{minted}{java}
boolean heuteDienstag = true;
if (heuteDienstag) {
  System.out.println("Crazyyy heute ist Dienstag");
}
      \end{minted}
          \pause
          \begin{ausgabe}
            Crazyyy heute ist Dienstag
          \end{ausgabe}
  \end{itemize}
\end{frame}

\begin{frame}[fragile]
  \frametitle{\texttt{if}-Verzweigung}
  \pause
  \begin{itemize}
    \item geht genauso mit \texttt{int} etc.
          \pause
    \item Beispiel:
          \begin{minted}{java}
int lieblingszahl = 42;
if (lieblingszahl == 42) {
  System.out.println("Du bist ein Highperformer!");
}
      \end{minted}
          \pause
          \begin{ausgabe}
            Du bist ein Highperformer!
          \end{ausgabe}
  \end{itemize}
\end{frame}



%IF-ELSE-VERZWEIGUNG
\begin{frame}[fragile]
  \frametitle{\texttt{if}-\texttt{else}}
  \pause
  \begin{itemize}
    \item erweitert die \texttt{if} Anweisung
          \pause
    \item "wenn \texttt{if} Bedingung nicht erfüllt dann mach folgendes\dots"
          \pause
          \begin{minted}{java}
if (Bedingung) {
  // Code bei Bedingung true
} else {
  // Code bei Bedingung false
}
\end{minted}
  \end{itemize}
\end{frame}

\begin{frame}[fragile]
  \frametitle{\texttt{if}-\texttt{else}}
  \pause
  \begin{itemize}
    \item Beispiel:
          \begin{minted}{java}
boolean heuteDonnerstag = false;
if (heuteDonnerstag) {
  System.out.println("endlich Wochenende");
} else {
  System.out.println(":( bestimmt ist bald wieder Donnerstag");
}
\end{minted}
          \pause
          \begin{ausgabe}
            :( bestimmt ist bald wieder Donnerstag
          \end{ausgabe}
  \end{itemize}
\end{frame}

\begin{frame}[fragile]
  \frametitle{\texttt{else if}}
  \pause
  \begin{itemize}
    \item \texttt{else if} prüft mehrere Bedingungen
          \pause
    \item Beispiel:
          \begin{minted}{java}
if (note == 1) {
    System.out.println("Sehr gut");
} else if (note == 2) {
    System.out.println("Gut");
} else if (note == 3) {
    System.out.println("Befriedigend");
} else {
    System.out.println("Ausreichend oder schlechter");
} 
    \end{minted}
  \end{itemize}
\end{frame}

\begin{frame}[plain]
  \centering
  {\Huge\bfseries{Check Yourself!}}
\end{frame}

\begin{frame}[fragile]
  \frametitle{Frage 1: Was ergibt...}
  \begin{minted}{java}
    false || true
  \end{minted}
  \pause
  \begin{ausgabe}
    true
  \end{ausgabe}
\end{frame}

\begin{frame}[fragile]
  \frametitle{Frage 2: Was ergibt...}
  \begin{minted}{java}
    true || false && false
  \end{minted}
  \pause
  \begin{ausgabe}
    true
  \end{ausgabe}
\end{frame}

\begin{frame}[fragile]
  \frametitle{Frage 3: Was wird ausgegeben ?}
  \begin{minted}{java}
    boolean a = true;
    boolean b = false;
    boolean result = a && b;
    System.out.println(result);
  \end{minted}
  \pause
  \begin{ausgabe}
    false
  \end{ausgabe}
\end{frame}

\begin{frame}[fragile]
  \frametitle{Frage 4: Was ergibt true?}
  \pause
  \begin{minted}{java}
    (true && (false || true)) || false
    // und auch 
    (true || false) && (true && true)
  \end{minted}
\end{frame}

\begin{frame}[fragile]
  \frametitle{Frage 5: Was wird ausgegeben?}
  \begin{minted}{java}
    boolean sonne = true;
    boolean regen = false;
    if (sonne && !regen) {
      System.out.println("Perfektes Wetter!");
    } else {
      System.out.println("Bleib lieber drinnen.");
    }
  \end{minted}
  \pause
  \begin{ausgabe}
    Perfektes Wetter!
  \end{ausgabe}
\end{frame}

\begin{frame}[fragile]
  \frametitle{Frage 6: Was wird ausgegeben?}
  \begin{minted}{java}
    int note = 3;
    if (note == 1) {
      System.out.println("Sehr gut");
    } else if (note == 2) {
      System.out.println("Gut");
    } else {
      System.out.println("Der Rest");
    }
  \end{minted}
  \pause
  \begin{ausgabe}
    Der Rest
  \end{ausgabe}
\end{frame}

\begin{frame}[fragile]
  \frametitle{Frage 7: Was wird ausgegeben?}
  \begin{minted}{java}
    boolean a = false;
    boolean b = true;
    if (a = b) {
      System.out.println("Wow!");
    } else {
      System.out.println("Nope!");
    }
  \end{minted}
  \pause
  \begin{ausgabe}
    Wow!
  \end{ausgabe}
    a = b ist eine Zuweisung, kein Vergleich!\\
\end{frame}

\begin{frame}[fragile]
  \frametitle{Frage 8: Was wird ausgegeben?}
  \begin{minted}{java}
    boolean a = true;
    boolean b = false;
    boolean c = true;
    if (a && (b || c)) {
      System.out.println("1");
    }
    if ((a && b) || (c && !b)){
      System.out.println("2");
    } 
    if ((a || b)) && !(c && b)) {
      System.out.println("3");
    }
  \end{minted}
  \pause
  \begin{ausgabe}
    1\newline
    2\newline
    3
  \end{ausgabe}
\end{frame}


%WHILE
\begin{frame}[fragile]
  \frametitle{\texttt{while}-Schleife}
  \pause
  \begin{itemize}
    \item  Wiederholt Anweisungen, solange eine Bedingung \texttt{true} ist
          \begin{minted}{java}
while (Bedingung) { /* Code */ }
        \end{minted}
  \end{itemize}
\end{frame}

\begin{frame}[fragile]
  \frametitle{\texttt{while}-Schleife}
  \pause
  \begin{minted}{java}
Scanner scanner = new Scanner(System.in);
String eingabe = "";

while (!eingabe.equals("ok")) {
  System.out.println("Bitte 'ok' eingeben:");
  eingabe = scanner.nextLine();
}
  \end{minted}
  \pause
  \begin{ausgabe}
    Bitte 'ok' eingeben: \newline
    ...\newline
    Bitte 'ok' eingeben: \newline
    (Benutzer tippt 'ok') → Schleife endet
  \end{ausgabe}
\end{frame}


%WHILE MIT BREAK
\begin{frame}[fragile]
  \frametitle{\texttt{break}}
  \pause
  \begin{itemize}
    \item Mit \texttt{break} kann man eine Schleife vorzeitig beenden
  \end{itemize}
  \pause
  \begin{minted}[fontsize=\footnotesize, linenos]{java}
Scanner scanner = new Scanner(System.in);

String passwort = "diegrillung";
String abbruchBedingung = "abbruch";
String momentaneEingabe = "";

while (!momentaneEingabe.equals(passwort)) {
    momentaneEingabe = scanner.nextLine();

    if (momentaneEingabe.equals(abbruchBedingung)) {
        break;
    }
}
System.out.println("I'm in");
\end{minted}
\end{frame}


%CODE TOGHETER
\begin{frame}[plain]
  \centering
  {\Huge\bfseries{Code together}}
\end{frame}



%FOR SCHLEIFE
\begin{frame}[fragile]
  \frametitle{Was ist eine \texttt{for}-Schleife?}
  \pause
  \begin{itemize}
    \item Wiederholt Anweisungen eine festgelegte Anzahl von Malen
          \pause
          \begin{minted}[fontsize=\large]{java}
for (Start; Bedingung; Schritt) {
    // Schleifenrumpf
}
\end{minted}
          \pause
    \item Ablauf:
          \pause
          \begin{enumerate}
            \item Wo starten wir? (z.B. bei 0)
                  \pause
            \item Bedingung noch erfüllt?
                  \pause
            \item Falls ja, Code im Block ausführen
                  \pause
            \item Zähler ändern i.d.R. einen hochzählen
          \end{enumerate}
  \end{itemize}
\end{frame}



%FOR BEISPIEL
\begin{frame}[fragile]
  \frametitle{\texttt{for}-Schleife}
  \pause
  Somit wird aus\dots

  \begin{minted}{java}
int momentaneWurst = 4;
int letzteWurst = 8;
while (momentaneWurst <= letzteWurst) {
    System.out.println("Schmeiß Wurst Nr." + momentaneWurst + " auf den Grill");
    momentaneWurst++;
}
  \end{minted}
  \pause
  ganz simpel\dots
  \begin{minted}{java}
for (int i = 4; i <= 8; i++) {
    System.out.println("Schmeiß Wurst Nr." + i + " auf den Grill");
}
  \end{minted}

\end{frame}

\end{document}
