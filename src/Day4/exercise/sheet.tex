\documentclass{../../sheet}
\renewcommand{\logopath}{../../logos/}

\title{PSE Vorkurs Tag 4}

\begin{document}
\maketitle

\newpage

\aufgabe{Aufgabe 1: Klassen}
Heute werden wir eine Klassenstruktur umsetzen, die ein Modell für eine Grillung sein soll.

\begin{enumerate}
    \item Erstelle zuallererst mit der Syntax aus der Präsi eine Grillklasse. Gib ihr einen Konstruktor, der ausgibt, dass ein neuer Grill erstellt wird. 
    \item In Java haben Klassen im Allgemeinen jeweils eine eigene Datei. Erstelle eine main-Klassen-Datei, sowie eine Grill-Klassen-Datei. Man kann jetzt in der Main in einer Funktion auf die Grillklasse zugreifen. Teste den Konstruktor der Grillklasse in dem du ein Grillobjekt in der Main-\textit{Funktion} der Main-\textit{Klasse} erstellst.
    \item Füge dem Grill Attribute hinzu, jeder Grill sollte einen Grillmeister, mit Namen als String benannt, eine maximale Würstchenanzahl, eine maximale Hitze, sowie eine momentane Hitze haben. Erweitere deinen Konstruktor so, dass auch die Attribute des Grills initialisiert werden, wenn ein neues Grillobjekt erstellt wird. Der Konstruktor soll den Namen des Grillmeisters als Parameter entgegennehmen.
    \item Gib dem Grill eine oder mehrere Funktionen zur Kontrolle der Hitze. Gestalte sie so wie du es für sinnvoll hältst.
    \item Füge dem Grill ein Array hinzu, dass die Würstchen symbolisieren soll. Erstelle zwei Funktionen zum Würstchen auf den Grill zu legen, sowie um sie wieder runterzunehmen. Die ''runternehmen''-Funktion sollte ausgeben, ob das Würstchen roh, durch oder verbrannt ist. Ein Würstchen wird mit dem Wert 0 auf den Grill gelegt. Ein Würstchen ist durch, wenn sein ''Durchheitswert'' zwischen 10 und 15 ist, wenn es über 15 gegrillt wird, dann ist es verbrannt. \\
    Füge dem Grill außerdem eine ''Grillen'' Funktion hinzu, die die momentane Hitze auf den Durchheitswert jeder Würstchen addiert.
    \item Nutze die main um einen Beispielhaften Grillablauf einer mehr oder weniger erfolgreichen Grillerei darzustellen. 
\end{enumerate}

\newpage
\aufgabe{Aufgabe 2: Klassen in Klassen}
Eine Sache, die dir vielleicht schon aufgefallen ist, ist dass wir momentan leider nur Würstchen Grillen können und auch bei denen nur genau eine Sorte. Dafür wird jetzt eine neue Grillgut-Klasse erstellt, die viel flexibler sein soll und unserer (leider digitalen) Grillung viel mehr ermöglichen wird.
\begin{enumerate}
    \item Erstelle eine neue Klasse (wieder mit neuer Datei) namens Grillgut. Sie soll einen Grillguttyp (String, z.B. Currywurst), einen minimalen Durchheitswert (int), einen maximalen Durchheitswert (int), einen momentanen Durchheitswert (int), eine Hitzetoleranz (int) und einen verbrannt (Boolean) Status haben. Wenn das Grillgut einmal mit einer Hitze größer der Hitzetoleranz gegrillt wird, dann ist es sofort verbrannt.
    \item Implementiere einen Konstruktor, der manche der obigen Attribute als Parameter entgegennimmt (die bei denen es dir sinnvoll erscheint), aber alle Attribute initialisiert, d.h. nach der Ausführung des Konstruktors sollten alle Attribute einen Wert haben.
    \item Erstelle eine grillen Funktion, die eine Hitze entgegennimmt und das Grillgut entsprechend Grillt. 
    \item Erstelle zudem eine vomGrillNehmen Funktion, die ausgibt, was gerade vom Grill genommen wurde und zudem in 5 Stufen ausgibt wie durch es ist:
    \begin{ausgabe}
Eine Currywurst ist fertig, sie ist noch roh.\\
Ein Steak ist fertig, es ist durch.\\
Ein Würstchen ist fertig, es ist gut durch.\\
Ein Spieß ist fertig, es ist leider verbrannt.\\
Ein Grillkäse ist fertig, er ist komplett Schwarz gebrannt.
    \end{ausgabe}
    \item Gehe zurück zu deiner Grillklasse und ersetze das \texttt{int[] aufDemGrill} Array mit einem \texttt{Grillgut[] aufDemGrill}. Mache als Folge dessen sinnvolle Anpassungen in der Grillklasse, die die Funktionen aus Grillgut verwenden.
    \item Nutze die main-Funktion um einen neuen Beispielhaften Grillablauf einer mehr oder weniger erfolgreichen
    Grillung darzustellen, aber jetzt mit vielen unterschiedlichen Grillgütern.
\end{enumerate}

\newpage
\aufgabe{HIGHPERFORMER: Optimale Grillerei/Grillgutvererbung}
Stellt euch vor, es war ein langer Tag im PSE-Vorkurs und ihr habt euch eine Grillung verdient, doch der Grill ist klein und es gibt viele unterschiedliche Grillgüter. Jetzt muss eine Methode her um möglichst schnell alles gegrillt zu bekommen und das ohne, dass ein einziges Stück verbrennt.


--


Grillgüter sind zum Glück noch viel vielfältiger, als das was wir hier darstellen. Vererbung für so Grillgut flippen oder drehen oder gar nichts. Spieße und Steaks und Würstchen. Gewürze auch in die Vererbungsstruktur rein. Maybe auch abstract.

\end{document}