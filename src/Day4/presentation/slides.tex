\documentclass{../../presentation}

\title{PSE - Vorkurs Tag 4}
\author{Linus, Philipp, Tillmann, Tobias}
\institute{FIUS - Fachgruppe Informatik Universität Stuttgart}
\date{09.10.2025}

\makeatletter
\renewcommand{\lecture@pathprefix}[1]{../../logos/}
\makeatother

\usepackage{todonotes}
\setuptodonotes{inline}

\begin{document}

\begin{frame}
	\titlepage
\end{frame}

\begin{frame}[fragile]
	\frametitle{Recap Tag 3 - Funktionen \& Scopes}

	\begin{itemize}
		\item\pause Funktionen helfen Code auszulagern und mehrfach zu nutzen
		      \begin{minted}[fontsize=\scriptsize]{java}
public static double avg(double a, double b, double c) {
    return (a + b + c) / 3;
}
          \end{minted}
		\item\pause \texttt{return} gibt einen Wert zurück und beendet die Funktion
		\item\pause Scopes: Variablen nur im jeweiligen Block sichtbar
		      \begin{minted}[fontsize=\scriptsize]{java}
int x = 5;
if (x > 0) {
    int y = 10; // y nur hier sichtbar
}
System.out.println(y); // Fehler!
          \end{minted}
	\end{itemize}
\end{frame}

\begin{frame}[fragile]
	\frametitle{Recap Tag 3 - Arrays}
	\pause
	\begin{itemize}
		\item\pause Arrays speichern mehrere Werte gleichen Typs
		      \begin{minted}[fontsize=\scriptsize]{java}
int[] zahlen = new int[5];
zahlen[0] = 42;
          \end{minted}
		\item\pause Zugriff über Index, Start bei \texttt{0}
		\item\pause Initialisierung auch direkt möglich:
		      \begin{minted}[fontsize=\scriptsize]{java}
String[] grill = {"Wurst", "Steak", "Mais"};
          \end{minted}
		\item Mit \texttt{.length} über alle Elemente iterieren:
		      \begin{minted}[fontsize=\scriptsize]{java}
for (int i = 0; i < grill.length; i++) {
    System.out.println(grill[i]);
}
          \end{minted}
	\end{itemize}
\end{frame}


\begin{frame}
	\frametitle{Angenommen \dots}
	\begin{itemize}
		\item\pause Wir wollen z.B. Informationen über Studierende speichern
		      \begin{itemize}
			      \item<2->[\textbullet] Name
			      \item<3->[\textbullet] Matrikelnummer
			      \item<4->[\textbullet] \dots
		      \end{itemize}
		\item<5-> Informationen verarbeiten
		      \begin{itemize}
			      \item<6->[\textbullet] testen ob Prüfung bestanden
			      \item<7->[\textbullet] \dots
		      \end{itemize}
	\end{itemize}
\end{frame}



% Melanie
\begin{frame}[fragile,t]
	\frametitle{Direkt umgesetzt}
	\vspace{2.0em}
	\begin{minipage}[t][0.6\textheight][t]{\textwidth}
		\begin{minted}{java}
String name1 = "Melanie";
int matrikelnummer1 = 12345;
\end{minted}
	\end{minipage}
\end{frame}



% Melanie + Paul
\begin{frame}[fragile,t]
	\frametitle{Direkt umgesetzt}
	\vspace{2.0em}
	\begin{minipage}[t][0.6\textheight][t]{\textwidth}
		\begin{minted}{java}
String name1 = "Melanie";
int matrikelnummer1 = 12345;

String name2 = "Paul";
int matrikelnummer2 = 54321;
\end{minted}
	\end{minipage}
\end{frame}



% komplett
\begin{frame}[fragile,t]
	\frametitle{Direkt umgesetzt}
	\vspace{2.0em}
	\begin{minipage}[t][0.6\textheight][t]{\textwidth}
		\begin{minted}{java}
String name1 = "Melanie";
int matrikelnummer1 = 12345;

String name2 = "Paul";
int matrikelnummer2 = 54321;

lernen(name1);
lernen(name2);

public static void lernen(String name) {
    System.out.println(name + " sagt: \"Ich hasse mein Leben\"");
}
\end{minted}
	\end{minipage}
	\only<2->{\begin{ausgabe}
			Melanie sagt: \texttt{"}Ich hasse mein Leben\texttt{"} \newline
			Paul sagt: \texttt{"}Ich hasse mein Leben\texttt{"}
		\end{ausgabe}}
\end{frame}



\begin{frame}[fragile]
	\frametitle{Warum so nicht?!}
	\begin{itemize}
		\item<2-> \textbf{Zusammengehörige Eigenschaften sind getrennt} \\
		      \texttt{name1}, \texttt{matriklenummer1}
		\item<3-> \textbf{Redundanz} \\
		      Immer wieder dieselben zwei Variablen - für jede Person
		\item<4-> \textbf{Kaum skalierbar} \\
		      Für 2 Studenten okay - aber bei 200? 20.000?
		\item<5-> \textbf{Keine klare Struktur} \\
		      Was genau ist ein Studierender im Code?
	\end{itemize}

	\vspace{2em}
	\centering
	\begin{minipage}{\textwidth}
		\centering
		\onslide<6->{\Huge $\rightarrow$~}%
		\onslide<6->{\Large Klassen schaffen Ordnung im Chaos}
	\end{minipage}
\end{frame}



% Teil-Folie
\begin{frame}[fragile,t]
	\frametitle{Was sind eigentlich Klassen?}

	\begin{minipage}[t][0.9\textheight][t]{\textwidth}
		\begin{itemize}
			\item eine Klasse ist ein \textbf{Bauplan} für Objekte
			\item<2->Eine Klasse definiert:
			      \begin{itemize}
				      \item[\textbullet] welche Eigenschaften ein Student hat (Attribute)
				      \item[\textbullet] was ein Student \texttt{"}kann\texttt{"} (Funktionen)
			      \end{itemize}
		\end{itemize}

		\vspace{3.5cm}
	\end{minipage}

\end{frame}



% Volle Folie
\begin{frame}[fragile,t]
	\frametitle{Was sind eigentlich Klassen?}

	\begin{minipage}[t][0.9\textheight][t]{\textwidth}
		\begin{itemize}
			\item eine Klasse ist ein \textbf{Bauplan} für Objekte
			\item Eine Klasse definiert:
			      \begin{itemize}
				      \item[\textbullet] welche Eigenschaften ein Student hat (Attribute)
				      \item[\textbullet] was ein Student \texttt{"}kann\texttt{"} (Funktionen)
			      \end{itemize}
		\end{itemize}

		\begin{minted}{java}
public class Student {
    // Eigenschaften (Attribute)
    String name;
    int matrikelnummer;

    // Verhalten (Funktionen)
    void lernen() {
        System.out.println(name + " sagt: \"Ich hasse mein Leben\"");
    }
}
\end{minted}
	\end{minipage}
\end{frame}

\begin{frame}[fragile]
	\frametitle{Klasse(n): Objekte}
	\begin{itemize}
		\item ein Objekt ist eine Instanz einer Klasse
		      \begin{itemize}
			      \item[\textbullet] z.B. Paul ist ein Objekt der Klasse Student
		      \end{itemize}
		\item<2-> Objekte werden mit einem \textbf{Konstruktor} erzeugt
		\item<3-> das ist der Konstruktor von \texttt{Student}:
		      \begin{minted}{java}
public Student(String name, int matrikelnummer) {
	this.name = name;
	this.matrikelnummer = matrikelnummer;
}
      \end{minted}
		      \begin{itemize}
			      \item<4->[\textbullet]dieser gibt vor wie das neue Objekt initialisiert wird
			      \item<5->[\textbullet]hat keinen Rückgabewert
			      \item<6->[\textbullet]heißt wie die Klasse
		      \end{itemize}
	\end{itemize}
\end{frame}

\begin{frame}[fragile]{Let\textquotesingle s build \texttt{paul}}
	\begin{itemize}
		\item Klasse inklusive Konstruktor:

		      \begin{minted}{java}
public class Student{
  String name;
  int matrikelnummer;

  public Student(String name, int matrikelnummer) {
    this.name = name;
    this.matrikelnummer = matrikelnummer;
  }

  void lernen() {
    System.out.println(name + " sagt: \"Ich hasse mein Leben\"");
  }
}
  \end{minted}
	\end{itemize}
\end{frame}

\begin{frame}[fragile]
	\frametitle{Let\textquotesingle s build \texttt{paul}}
	\begin{itemize}
		\item Klasse mit Konstruktor (gekürzt, nicht kompilierbar):
		      \begin{minted}{java}
// ... Attribute 
public Student(String name, int matrikelnummer) {
  this.name = name;
  this.matrikelnummer = matrikelnummer;
}
// ... Funktionen
			  \end{minted}
		\item<2->\texttt{new Student(\dots)} ruft den oben erstellten Konstruktor auf:
		      \begin{minted}{java}
Student paul = new Student("Paul", 12345);
			  \end{minted}
	\end{itemize}
\end{frame}


\begin{frame}[fragile]
	\frametitle{\texttt{paul} ein Highperformer}
	\begin{minted}{java}
// Objekt erzeugen
Student paul = new Student("Paul", 12345);

// Attribute auslesen
System.out.println(paul.name);
System.out.println(paul.matrikelnummer);

// Funktion aufrufen
paul.lernen();
\end{minted}

	\begin{ausgabe}
		Paul

		12345

		Paul sagt: \texttt{"}Ich hasse mein Leben\texttt{"}
	\end{ausgabe}
\end{frame}

\begin{frame}[fragile]
	\frametitle{Wie sehen Klassen in Java aus?}
	\pause
	\begin{itemize}
		\item Klassen werden in eigenen Dateien gespeichert
		      \pause
		      \begin{itemize}
			      \item[\textbullet] Dateiname = Klassenname + \texttt{.java}
		      \end{itemize}
		      \pause
		\item Beispiel: \texttt{Student.java} enthält die Klasse \texttt{Student}
		      \pause
		\item Attribute und Funktionen werden innerhalb der Klasse definiert
	\end{itemize}
\end{frame}

\begin{frame}[fragile]
	\frametitle{Sichtbarkeit in Klassen}
	\pause
	\begin{itemize}
		\item \texttt{private}: nur innerhalb der Klasse sichtbar
		      \pause
		\item \texttt{public}: überall sichtbar (auch von außen)
		      \pause
		      \begin{minted}{java}
public class Student {
    private String name;         
    public int matrikelnummer;   

    public void lernen() {
        System.out.println(name + " sagt: \"Ich hasse mein Leben\"");
    }
}
	\end{minted}
	\end{itemize}
\end{frame}



\begin{frame}[fragile]
	\frametitle{Sichtbarkeit in Klassen}
	\pause
	\begin{itemize}
		\item\texttt{matrikelnummer} und \texttt{lernen} ist \texttt{public} → kann man von außen verwenden
		      \begin{minted}{java}
Student paul = new Student("Paul", 12345);

System.out.println(paul.matrikelnummer);
paul.lernen();
  \end{minted}
		      \pause
		      {\begin{ausgabe}
				      12345


				      Paul sagt: \texttt{"}Ich hasse mein Leben\texttt{"}
			      \end{ausgabe}}
		      \pause
		\item \texttt{name} ist \texttt{private} → kann nur innerhalb der Klasse verwendet werden
		      \begin{minted}{java}
Student paul = new Student("Paul", 12345);

System.out.println(paul.name);
  \end{minted}
		      \pause
		      \begin{ausgabe}
			      \alert{java: name hat private-Zugriff in Student}
		      \end{ausgabe}
	\end{itemize}
\end{frame}


\begin{frame}[fragile]
	\frametitle{Attribute können auch \texttt{final} sein}

	\begin{itemize}
		\item<2-> mit \texttt{final} kann Attribut nach Initialisierung nicht mehr geändert werden
		      \begin{minted}{java}
<Sichtbarkeit> final <Datentyp> <Name> = <Wert>;
		\end{minted}
		\item<3-> Beispiel:
		      \begin{minted}{java}
public class Student {
    private final String name;
    public int matrikelnummer;
    
  public Student(String name, int matrikelnummer) {
    this.name = name; //kann jetzt nie wieder geändert werden
    this.matrikelnummer = matrikelnummer; //kann beliebig oft geändert werden
  }
}
\end{minted}

		\item<4-> Wird versucht \texttt{name} später zu ändern → \alert{Fehler}
	\end{itemize}

\end{frame}


\begin{frame}[plain]
	\centering
	{\Huge\bfseries{Fuiz los geht's!}}
\end{frame}

\begin{frame}[fragile]
	\frametitle{Frage 1: Wie nennt man ein Objekt, das von einer Klasse erstellt wurde?}
	\begin{ausgabe}
		Konstruktor
	\end{ausgabe}
	\begin{ausgabe}
		\textcolor<2->{green!100!black}{Instanz}
	\end{ausgabe}
	\begin{ausgabe}
		Attribut
	\end{ausgabe}
	\begin{ausgabe}
		Funktion
	\end{ausgabe}

\end{frame}

\begin{frame}[fragile]
	\frametitle{Frage 2: Was beschreibt Klassen am besten?}
	\begin{ausgabe}
		Eine fertige Instanz von Daten
	\end{ausgabe}
	\begin{ausgabe}
		Eine Gruppe von ähnlichen Funktionen
	\end{ausgabe}
	\begin{ausgabe}
		\textcolor<2->{green!100!black}{Ein Bauplan für Objekte}
	\end{ausgabe}
	\begin{ausgabe}
		Eine Gruppe von Variablen
	\end{ausgabe}
\end{frame}

\begin{frame}[fragile]
	\frametitle{Frage 3: Welche Aussage trifft auf Konstruktoren zu?}
	\begin{ausgabe}
		Ein Konstruktor hat immer den Rückgabetyp void
	\end{ausgabe}
	\begin{ausgabe}
		\textcolor<2->{green!100!black}{Ein Konstruktor initialisiert Objekte}
	\end{ausgabe}
	\begin{ausgabe}
		Ein Konstruktor kann beliebig benannt werden
	\end{ausgabe}
	\begin{ausgabe}
		Konstruktoren dürfen keine Parameter haben
	\end{ausgabe}
\end{frame}

\begin{frame}[fragile]
    \frametitle{Frage 4: Eine Funktion innerhalb einer Klasse kann auf alle Attribute dieser Klasse zugreifen.}

    \begin{ausgabe}
        \textcolor<2->{green!100!black}{wahr}
    \end{ausgabe}

    \begin{ausgabe}
        falsch
    \end{ausgabe}
\end{frame}

\begin{frame}[fragile]
	\frametitle{Frage 5: Was ist notwendig für eine Klasse?}
	\begin{ausgabe}
		\textcolor<2->{green!100!black}{Ein Bezeichner}
	\end{ausgabe}
	\begin{ausgabe}
		Attribute
	\end{ausgabe}
	\begin{ausgabe}
		Ein definierter Konstruktor
	\end{ausgabe}
	\begin{ausgabe}
		Funktionen
	\end{ausgabe}
\end{frame}

\begin{frame}[fragile]
	\frametitle{Frage 6: Welche Aussage ist wahr?}
	\begin{minted}{java}
public class Student {
    private final String name;
    public int matrikelnummer;

    public Student(String name, int matrikelnummer) {
        this.name = name;
        this.matrikelnummer = matrikelnummer;
    }

    public void lernen() {
        System.out.println(name + " sagt: \"Ich hasse mein Leben\"");
    }
}
	\end{minted}
	\begin{ausgabe}
		\textcolor{green!100!black}{Name kann nur im Konstruktor gesetzt werden}
	\end{ausgabe}
\end{frame}

\begin{frame}[fragile]
	\frametitle{Frage 7: Was bedeutet es wenn ein Attribut private ist?}
	\begin{ausgabe}
		\textcolor<2->{green!100!black}{Es kann nur innerhalb der Klasse darauf zugegriffen und verändert werden}
	\end{ausgabe}
	\begin{ausgabe}
		Es kann nur innerhalb der Klasse Verändert werden
	\end{ausgabe}
	\begin{ausgabe}
		Es kann nur innerhalb der Klasse darauf zugegriffen werden
	\end{ausgabe}
	\begin{ausgabe}
		Es kann nicht verändert werden
	\end{ausgabe}
\end{frame}

\begin{frame}[fragile]
	\frametitle{Klassen in Klassen}
	\pause
	\begin{itemize}
		\item Klassen können auch Objekte von anderen Klassen enthalten
		      \pause
		\item Beispiel: \texttt{Universitaet} enthält viele \texttt{Studenten} (Objekte der \texttt{Student} Klasse)
		      \pause
		      \begin{minted}{java}
public class Universitaet {
    Student[] alleStudis;

    public Universitaet(int anzahl) {
        alleStudis = new Student[anzahl];
        for (int i = 0; i < anzahl; i++) {
            alleStudis[i] = new Student("Paul", i);
        }
    }
}
\end{minted}
	\end{itemize}
\end{frame}

\begin{frame}[fragile]
	\frametitle{Alle Studenten lernen}

	\begin{itemize}
		\item<2-> Neue Methode in \texttt{Universitaet}:
		      \begin{minted}{java}
public void pruefungsvorbereitung() {
	for (Student s : alleStudis) {
		s.lernen();
	}
}
\end{minted}
	\end{itemize}
\end{frame}
\begin{frame}[fragile]
	\frametitle{so schnell kanns gehen}

	\begin{itemize}
		\item<2-> Was passiert hier?
		      \begin{minted}{java}
Universitaet uniStuttgart = new Universitaet(3);
uniStuttgart.pruefungsvorbereitung();
\end{minted}
		      \begin{itemize}
			      \item<3->[\textbullet]uniStuttgart wird mit anzahl 3 erstellt
			      \item<4->[\textbullet]der \texttt{Universitaet} Konstruktor erstellt 3 Studenten
			      \item<5->[\textbullet]durch \texttt{pruefungsvorbereitung} lernen alle Studis
		      \end{itemize}
		      \only<6->{
			      \begin{ausgabe}
				      Paul sagt: \texttt{"}Ich hasse mein Leben\texttt{"}

				      Paul sagt: \texttt{"}Ich hasse mein Leben\texttt{"}

				      Paul sagt: \texttt{"}Ich hasse mein Leben\texttt{"}
			      \end{ausgabe}}
	\end{itemize}
\end{frame}

\renewcommand{\feedbackurl}{https://forms.gle/JasKFVgR2ARoFVTEA}
\begin{frame}[fragile]
    \frametitle{Ende}
    \begin{columns}
        \begin{column}{0.65\textwidth}
            \begin{itemize}
                \item Folien und Aufgaben: \newline \url{https://fius.de/index.php/pse-vk-folien/}
            \end{itemize}
        \end{column}
        \begin{column}{0.35\textwidth}
            \qrcode{https://fius.de/index.php/pse-vk-folien/}
        \end{column}
    \end{columns}

    \vspace{3em}

    \begin{columns}
        \begin{column}{0.65\textwidth}
            \begin{itemize}
                \item Feedback: \newline \url{\feedbackurl}
            \end{itemize}
        \end{column}
        \begin{column}{0.35\textwidth}
            \qrcode{\feedbackurl}
        \end{column}
    \end{columns}

    Wenn ihr Fragen habt, sagt Bescheid!

\end{frame}

\end{document}